\section{Evaluation}

We build \tool upon Sysdig~\cite{sysdig}, and deployed our tool in a server to collect system auditing events and perform attack investigation. 
The server is used by other users for performing daily tasks, so that enough noise of irrelevant system activities can be collected.
We performed a series of attacks based on known exploits in the deployed environment,
and applied \tool to perform attack investigation on the attacks, demonstrating the effectiveness of \tool.
In total, our evaluations use 
%53GB of 
real
%52GB of real, 24-hour unstopped, 
system monitoring data that consists of \emph{2 billion} events. 
Each attack is done with the time gap being at least 1 hour.

Specifically, we conduct three sets of evaluations.
First, to evaluate the effectiveness of \tool in propagating reputations,
we compare the reputation scores of the POI entities against the expected reputation scores in benign scenarios (POI entities coming from trusted sources) and attack scenarios (POI entities coming from suspicious sources).
We also compare the results with three other weight computation approaches.
Second, we compare the weight computation in \tool with the edge prioritization of the state-of-the-art causality analysis, PrioTracker~\cite{liu2018priotracker}, which prioritizes edges during dependency search based on the fanout of nodes.
Finally, we evaluate the effectiveness of \tool in revealing critical edges for attack sequence reconstruction.





% \subsection{Evaluation Results}
% To evaluate the accuracy and effect of our method, we prepared 20 cases. There are 10 cases covering the most common user activities including file read/write, network download and decompress. Table~\ref{tab:benignHighRP} shows the reputation result of four methods.
% Table~\ref{tab:Reduction} shows our method reduction rate.

\subsection{Overall Evaluation Setup}
\label{subsec:cases}
The evaluations are conducted on a server with an Intel(R) Xeon(R) CPU E5-2637 v4 (3.50GHz), 256GB RAM running 64bit Ubuntu 18.04.1.
We performed 8 tasks to inject benign and malicious payloads into the system through key system interfaces that are vulnerable for attacks.
We also performed 5 real APT attacks in the deployed environment. 
We then collected the system auditing events and applied \tool to analyze the events.


\subsubsection{Benign and Malicious Payloads Through Key System Interfaces}
\label{subsub:benign-cases}
We performed 8 tasks that employ the common system interfaces to inject benign or malicious payloads. These representative system interfaces are commonly exploited in attacks~\cite{securitybook}.


\begin{itemize}[noitemsep, topsep=1pt, partopsep=1pt, listparindent=\parindent, leftmargin=*]
\item File merge: \emph{2File}, \emph{3File}

\item Shell execution: \emph{shell-script} (list all files in the Home folder and write the results to a file)

\item File download: \emph{curl}, \emph{wget}, \emph{shell-wget} (wget called by a shell script), \emph{python-wget} (wget called by a Python script)

\item File transfer: \emph{scp}
\end{itemize}

\begin{table*}[]
        \centering
        \caption{Reputation Results Of Representative Cases for Key System Interfaces (seeds RP = 1.0)}
        \label{tab:benignHighRP}
        \resizebox{0.9\textwidth}{!}{%
        \begin{tabular}{|l|r|r|r|r|r|r|r|}
        \hline
        \thead{Case}                 & \thead{ManualProjection} & \thead{GlobalProjection} & \thead{Comparison(\%)} & \thead{GlobalProjectionNoOutlier} & \thead{Comparison(\%)} & \thead{LocalProjection (\tool)} & \thead{Comparison(\%)} \\ \hline
        3File             & 8.51E-01           & 5.39E-01    & -36.64         & 1.00E+00                & 17.44          & 1.00E+00      & 17.44          \\ \hline
        2File             & 8.02E-01           & 5.27E-01    & -34.26         & 1.00E+00                & 24.70          & 1.00E+00      & 24.70          \\ \hline
        USB-merge         & 9.18E-01           & 9.93E-01    & 8.16           & 9.99E-01                & 8.75           & 9.79E-01      & 6.64           \\ \hline
        shell-script       & 6.50E-01           & 7.16E-01    & 10.18          & 9.02E-01                & 38.69          & 9.46E-01      & 45.57          \\ \hline
        curl              & 9.15E-01           & 9.21E-01    & 0.68           & 9.85E-01                & 7.65           & 1.00E+00      & 9.25           \\ \hline
        wget              & 9.82E-01           & 9.83E-01    & 0.12           & 1.00E+00                & 1.86           & 1.00E+00      & 1.85           \\ \hline
        python-wget       & 8.05E-01           & 6.51E-01    & -19.09         & 8.82E-01                & 9.55           & 9.99E-01      & 24.08          \\ \hline
        python-wget-unzip & 6.47E-01           & 5.63E-01    & -12.99         & 7.16E-01                & 10.58          & 7.57E-01      & 16.88          \\ \hline
        scp               & 6.85E-01           & 8.80E-01    & 28.33          & 9.69E-01                & 41.30          & 9.58E-01      & 39.78          \\ \hline
        shell-wget        & 8.00E-01           & 9.10E-01    & 13.83          & 7.56E-01                & -5.49          & 9.58E-01      & 19.84          \\ \hline
        shell-wget-unzip  & 6.81E-01           & 7.45E-01    & 9.38           & 8.21E-01                & 20.59          & 9.37E-01      & 37.58          \\ \hline
        \thead{average}   & 7.94E-01           & 7.66E-01    & -2.94          & 9.12E-01                & 15.97          & 9.58E-01      & 22.15          \\ \hline
        \end{tabular}
        \label{tab:normalcase}
        }
\end{table*}
% \begin{table*}[]
        \centering
        \caption{Reputation Results Of Key Steps in APT Attacks (seeds RP = 1.0)}
        \label{tab:attackHighRP}
        \resizebox{0.9\textwidth}{!}{%
        \begin{tabular}{|l|r|r|r|r|r|r|r|}
        \hline
        \thead{Case}                 & \thead{ManualProjection} & \thead{GlobalProjection} & \thead{Comparison(\%)} & \thead{GlobalProjectionNoOutlier} & \thead{Comparison(\%)} & \thead{LocalProjection (\tool)} & \thead{Comparison(\%)} \\ \hline
        command-injection-c1 & 8.90E-01           & 9.65E-01    & 8.39           & 9.98E-01                & 12.07          & 9.98E-01      & 12.07          \\ \hline
        command-injection-c2 & 7.00E-01           & 5.00E-01    & -28.53         & 9.69E-01                & 38.36          & 9.92E-01      & 41.64          \\ \hline
        data-leakage         & 6.76E-01           & 7.22E-01    & 6.82           & 7.95E-01                & 17.67          & 1.00E+00      & 47.91          \\ \hline
        password-crack-c1    & 9.40E-01           & 9.74E-01    & 3.63           & 1.00E+00                & 6.41           & 1.00E+00      & 6.41           \\ \hline
        password-crack-c2    & 9.11E-01           & 9.63E-01    & 5.72           & 9.88E-01                & 8.52           & 1.00E+00      & 9.78           \\ \hline
        password-crack-c3    & 9.89E-01           & 8.57E-01    & -13.42         & 1.00E+00                & 1.06           & 9.99E-01      & 1.01           \\ \hline
        penetration-c1        & 8.83E-01           & 9.92E-01    & 12.31          & 9.92E-01                & 12.31          & 9.67E-01      & 9.45           \\ \hline
        penetration-c2        & 7.95E-01           & 8.53E-01    & 7.23           & 7.88E-01                & -0.87          & 9.47E-01      & 19.10          \\ \hline
        vpnfilter-c1         & 9.55E-01           & 9.75E-01    & 2.11           & 9.92E-01                & 3.92           & 9.92E-01      & 3.92           \\ \hline
        vpnfilter-c2         & 9.67E-01           & 9.99E-01    & 3.34           & 9.99E-01                & 3.36           & 9.99E-01      & 3.36           \\ \hline
        \thead{average}      & 8.71E-01           & 8.80E-01    & 0.76           & 9.52E-01                & 10.28          & 9.89E-01      & 15.47          \\ \hline
        \end{tabular}
        }
\end{table*}
\begin{table}[!tb]
        \centering
        \caption{POI reputations of malicious payloads through key system interfaces (expected: $0.0$)}
        \label{tab:benignLowRP}
        \resizebox{0.48\textwidth}{!}{%
            \begin{tabular}{|l|r|r|r|r|r|}
           \hline
            \multicolumn{1}{|c|}{\textbf{Attack/Task}} & \multicolumn{1}{c|}{\textbf{\lpfixed}} & \multicolumn{1}{c|}{\textbf{PrioTracker}} & \multicolumn{1}{c|}{\textbf{\lpglobal}} & \multicolumn{1}{c|}{\textbf{\lpglobalplus}} & \multicolumn{1}{c|}{\textbf{\tool}} \\ \hline
            3File                               & 0.15                                   & 0.50                                 & 0.49                                   & 0.02                                             & $\sim$0.00                                  \\ \hline
            2File                               & 0.20                                   & 0.50                                 & 0.50                                   & 0.50                                             & $\sim$0.00                                  \\ \hline
            curl                                & 0.30                                   & 0.40                                 & 0.49                                   & 0.49                                             & 0.01                                  \\ \hline
            shell\_script                       & 0.38                                   & 0.50                                 & 0.50                                   & 0.50                                             & 0.04                                  \\ \hline
            python\_wget                        & 0.29                                   & 0.35                                 & 0.39                                   & 0.37                                             & 0.01                                  \\ \hline
            scp                                 & 0.26                                   & $\sim$0.00                                 & 0.07                                   & 0.08                                             & $\sim$0.00                                  \\ \hline
            shell\_wget                         & 0.20                                   & 0.25                                 & 0.18                                   & 0.11                                             & 0.02                                  \\ \hline
            wget                                & 0.21                                   & 0.46                                 & 0.50                                   & 0.50                                             & $\sim$0.00                                  \\ \hline
            \textbf{avg}                                 & 0.25                                   & 0.37                                 & 0.39                                   & 0.32                                             & 0.01                                  \\ \hline
            \end{tabular}
        }
\end{table}

\subsubsection{Real Attacks}
\label{subsubsec:attack-cases}

Besides common exploits, we performed 3 real attacks that capture the important traits of attacks depicted from the Cyber Kill Chain framework~\cite{cyberkillchain}. 
Note that theses attacks consist of a series of steps, and some steps may not be captured by system auditing (\eg user inputs and inter-process communications).
Such limitations can be addressed by employing more powerful auditing tools, which is out of the scope of this paper.
\eat{
\myparatight{Attack 1: Zero-Day Penetration to Target Host}
The scenario emulates the attacker's behavior who penetrates the victim's host
leveraging previously unknown Zero-day attack. Zero-day vulnerabilities are
attack vectors that previously unknown to the community, therefore allow the
attacker to put their first step into their targets. In our case, we assume that
the {\tt bash} binary in victim's host is outdated and vulnerable to shellshock~\cite{shellshock}. The victim computer hosts web service that has
CGI written as BASH script. The attacker can run an arbitrary command when she
passes the specially crafted attack string as one of environment variable. Leveraging the vulnerability, the attacker runs a series of remote commands to
plant and run initial attack by: (1) transferring the payload (\emph{penetration-c1}), (2) changing its permission, and (3) running the payload to bootstrap its campaign (\emph{penetration-c2}).
% As a lateral movement, the
% attacker downloads (4) a password cracker program from outside run it against
% the shadow password files. 
}


\myparatight{Attack 1: Password Cracking After Shellshock Penetration}
% Once breaks into the system, the attacker can launch different malicious
% behaviors (\eg password cracking, information leakage, denial of services). 
After the initial shellshock penetration, the attacker first connects to Cloud services (\eg Dropbox, Twitter) and downloads an image where C2 (Command and Control) host's IP address is encoded in the EXIF metadata (\emph{password-crack-c1}). The behavior is a common practice shared by APT attacks~\cite{hammertoss,vpnfilter} to evade the network-based detection system based on DNS blacklisting.

Using the IP, the malware connects to the C2 host. 
The C2 host directs the malware to take some lateral movements, including a series of stealthy reconnaissance maneuvers. 
In this stage, the attacker generally takes a number of actions. 
Among those, we emulate the password cracking attack. 
The attacker downloads password cracker payload  and runs it against password shadow files (\emph{shellshock}).

\myparatight{Attack 2: Data Leakage After Shellshock Penetration}
After the lateral movement stage, the attacker attempts to steal all the valuable assets from the host. 
This stage mainly involves the behaviors of local and remote file system scanning activities, copying and compressing of important files, and transferring the files to its C2 host.
The attacker scans the file system, scrap files into a single compress file and transfer it back to C2 host (\emph{data-leakage}).

\eat{
\myparatight{Attack 4: Command-line Injection with Input Sanitization Failures}
Different from the previous shellshock case, a program may contain vulnerabilities introduced by developer errors and this can also be a initial attack vector that invites the attacker into their target systems. To represent
such cases, we wrote an web application prototype that fails to sanitize inputs for a certain web request, hence allows Command line Injection attack. 
Our prototype service mimics the Jeep-Cherokee attack case~\cite{miller:remote:2015} which implements a remote access using the conventional web service API that
internally uses DBUS service to run the designated commands. 
Due to the developers' mistake, the web service fails to sanitize the remote inputs, the attacker can append arbitrary commands followed by semi-colon({\tt;}). 
Leveraging this vulnerability, we can download backdoor program (\emph{commend-injection-c1}) and collect sensitive data (\emph{command-injection-c2}).
}

\myparatight{Attack 3: VPNFilter}
We prototyped a famous IoT attack campaign: VPNFilter malware~\cite{vpnfilterschenier}, which infected millions of IoT devices by exploiting a number of known or zero-day vulnerabilities~\cite{vpnfilter1,vpnfilter2}. 
The attack's significance lies in how the malware operates during its lateral movement stage following its initial penetration. 
The campaign employs up-to-date hacker practices to bypass conventional security solutions based on static blacklisting approaches and has an architecture to download the plug-in payload on-demand at run-time. 
We prototyped the malware by referring to one of its sample for x86 architecture~\cite{vpnfilterx86}.

The VPNFilter stage 1 malware accesses a public image repository to get an image. In the EXIF metadata of the image, it contains the IP address for the stage 2 host. It downloads the VPNFilter stage2 from the stage2 server, and executes it to launch the attack (\emph{vpnfilter}).








\subsection{Evaluation Results}
\label{subsec:eval-results}







\subsubsection{Reputation Propagation}
\label{subsec:reputation-results}
We evaluate the effectiveness of \tool in identifying whether POI entities come from trusted sources or untrusted sources via reputation propagation in both normal and malicious scenarios, respectively.

\myparatight{Reputation Assignment}
% All the nodes that have no incoming edges are considered as seed nodes.
For evaluation purposes, 
we set the reputation of entry nodes representing trusted sources to $1.0$,
and entry nodes representing system libraries to $0.5$.
In malicious scenarios and real attacks, we set the reputation of entry nodes representing untrusted sources to $0.0$.
We propagate the reputation from entry nodes,
and record the reputation scores of the POI (referred to as \emph{POI reputation}).
An effective approach will lead to a POI reputation that is close to $1.0$ in the benign scenarios, instead of closing to $0.5$ (similar to neutral libraries) or $0.0$ (incorrectly associating the POI to untrusted sources).
Similarly, an effective approach should lead to a POI reputation close to $0.0$ in the malicious scenarios.

\myparatight{Comparison Approaches}
To demonstrate the effectiveness of \tool's weight computation, we compare \tool with the following three weight computation approaches:
\begin{itemize}[noitemsep, topsep=1pt, partopsep=1pt, listparindent=\parindent, leftmargin=*]
    
    \item \lpfixed: We select a fixed parameter vector $(0.1, 0.5, 0.4)$ by manually tuning the parameters and then normalize it to be the projection vector.
    
    \item \lpglobal: We globally cluster all edges in the graph using Multi-KMeans++ and compute the projection vector using our extended version of LDA. 
    
    \item \lpglobalplus: Same as previous one, but for nodes that have only one incoming edge (\ie outlier edges), we do not consider these edges in the global clustering and global projection vector computation, and directly assign their final weights to $1.0$.

    % \item \tool: This is the one described in \cref{subsec:weight-computation}. We locally cluster the incoming edges of every node using Multi-KMeans++ and locally compute the projection vector using extended LDA.
    
\end{itemize}

Furthermore, we compare our weight computation approach with the fanout approach used in the state-of-the-art causality analysis approach, PrioTracker~\cite{liu2018priotracker}.
PrioTracker mainly uses the fanout of nodes to prioritize the dependencies in the causality analysis for a given POI event.
% While they also use a rareness score based on the reference models built upon normal activities in their proprietary environment, their models are not publicly available and difficult to generalize from their organizations to our deployed environment.
% Given that the rareness score accounts for only a small portion of their priority (\ie 27\%), we use only the fanout of nodes to compute edge priories.\footnote{Note that the reference models are complementary to \tool, and \tool can easily integrate the rareness score with our weights to compute to final edge weights.}
We then adapt the computed priories as the edge weights, and apply our algorithm for reputation propagation.
In this way, we can do a fair comparison between \tool and fanout approach in reputation propagation.
However, note that PrioTracker does not enable reputation propagation as \tool does.

\begin{table*}[]
        \centering
        \caption{Reputation Results Of Representative Cases for Key System Interfaces (seeds RP = 1.0)}
        \label{tab:benignHighRP}
        \resizebox{0.9\textwidth}{!}{%
        \begin{tabular}{|l|r|r|r|r|r|r|r|}
        \hline
        \thead{Case}                 & \thead{ManualProjection} & \thead{GlobalProjection} & \thead{Comparison(\%)} & \thead{GlobalProjectionNoOutlier} & \thead{Comparison(\%)} & \thead{LocalProjection (\tool)} & \thead{Comparison(\%)} \\ \hline
        3File             & 8.51E-01           & 5.39E-01    & -36.64         & 1.00E+00                & 17.44          & 1.00E+00      & 17.44          \\ \hline
        2File             & 8.02E-01           & 5.27E-01    & -34.26         & 1.00E+00                & 24.70          & 1.00E+00      & 24.70          \\ \hline
        USB-merge         & 9.18E-01           & 9.93E-01    & 8.16           & 9.99E-01                & 8.75           & 9.79E-01      & 6.64           \\ \hline
        shell-script       & 6.50E-01           & 7.16E-01    & 10.18          & 9.02E-01                & 38.69          & 9.46E-01      & 45.57          \\ \hline
        curl              & 9.15E-01           & 9.21E-01    & 0.68           & 9.85E-01                & 7.65           & 1.00E+00      & 9.25           \\ \hline
        wget              & 9.82E-01           & 9.83E-01    & 0.12           & 1.00E+00                & 1.86           & 1.00E+00      & 1.85           \\ \hline
        python-wget       & 8.05E-01           & 6.51E-01    & -19.09         & 8.82E-01                & 9.55           & 9.99E-01      & 24.08          \\ \hline
        python-wget-unzip & 6.47E-01           & 5.63E-01    & -12.99         & 7.16E-01                & 10.58          & 7.57E-01      & 16.88          \\ \hline
        scp               & 6.85E-01           & 8.80E-01    & 28.33          & 9.69E-01                & 41.30          & 9.58E-01      & 39.78          \\ \hline
        shell-wget        & 8.00E-01           & 9.10E-01    & 13.83          & 7.56E-01                & -5.49          & 9.58E-01      & 19.84          \\ \hline
        shell-wget-unzip  & 6.81E-01           & 7.45E-01    & 9.38           & 8.21E-01                & 20.59          & 9.37E-01      & 37.58          \\ \hline
        \thead{average}   & 7.94E-01           & 7.66E-01    & -2.94          & 9.12E-01                & 15.97          & 9.58E-01      & 22.15          \\ \hline
        \end{tabular}
        \label{tab:normalcase}
        }
\end{table*}
% \begin{table*}[]
        \centering
        \caption{Reputation Results Of Key Steps in APT Attacks (seeds RP = 1.0)}
        \label{tab:attackHighRP}
        \resizebox{0.9\textwidth}{!}{%
        \begin{tabular}{|l|r|r|r|r|r|r|r|}
        \hline
        \thead{Case}                 & \thead{ManualProjection} & \thead{GlobalProjection} & \thead{Comparison(\%)} & \thead{GlobalProjectionNoOutlier} & \thead{Comparison(\%)} & \thead{LocalProjection (\tool)} & \thead{Comparison(\%)} \\ \hline
        command-injection-c1 & 8.90E-01           & 9.65E-01    & 8.39           & 9.98E-01                & 12.07          & 9.98E-01      & 12.07          \\ \hline
        command-injection-c2 & 7.00E-01           & 5.00E-01    & -28.53         & 9.69E-01                & 38.36          & 9.92E-01      & 41.64          \\ \hline
        data-leakage         & 6.76E-01           & 7.22E-01    & 6.82           & 7.95E-01                & 17.67          & 1.00E+00      & 47.91          \\ \hline
        password-crack-c1    & 9.40E-01           & 9.74E-01    & 3.63           & 1.00E+00                & 6.41           & 1.00E+00      & 6.41           \\ \hline
        password-crack-c2    & 9.11E-01           & 9.63E-01    & 5.72           & 9.88E-01                & 8.52           & 1.00E+00      & 9.78           \\ \hline
        password-crack-c3    & 9.89E-01           & 8.57E-01    & -13.42         & 1.00E+00                & 1.06           & 9.99E-01      & 1.01           \\ \hline
        penetration-c1        & 8.83E-01           & 9.92E-01    & 12.31          & 9.92E-01                & 12.31          & 9.67E-01      & 9.45           \\ \hline
        penetration-c2        & 7.95E-01           & 8.53E-01    & 7.23           & 7.88E-01                & -0.87          & 9.47E-01      & 19.10          \\ \hline
        vpnfilter-c1         & 9.55E-01           & 9.75E-01    & 2.11           & 9.92E-01                & 3.92           & 9.92E-01      & 3.92           \\ \hline
        vpnfilter-c2         & 9.67E-01           & 9.99E-01    & 3.34           & 9.99E-01                & 3.36           & 9.99E-01      & 3.36           \\ \hline
        \thead{average}      & 8.71E-01           & 8.80E-01    & 0.76           & 9.52E-01                & 10.28          & 9.89E-01      & 15.47          \\ \hline
        \end{tabular}
        }
\end{table*}
\begin{table}[!tb]
        \centering
        \caption{POI reputations of malicious payloads through key system interfaces (expected: $0.0$)}
        \label{tab:benignLowRP}
        \resizebox{0.48\textwidth}{!}{%
            \begin{tabular}{|l|r|r|r|r|r|}
           \hline
            \multicolumn{1}{|c|}{\textbf{Attack/Task}} & \multicolumn{1}{c|}{\textbf{\lpfixed}} & \multicolumn{1}{c|}{\textbf{PrioTracker}} & \multicolumn{1}{c|}{\textbf{\lpglobal}} & \multicolumn{1}{c|}{\textbf{\lpglobalplus}} & \multicolumn{1}{c|}{\textbf{\tool}} \\ \hline
            3File                               & 0.15                                   & 0.50                                 & 0.49                                   & 0.02                                             & $\sim$0.00                                  \\ \hline
            2File                               & 0.20                                   & 0.50                                 & 0.50                                   & 0.50                                             & $\sim$0.00                                  \\ \hline
            curl                                & 0.30                                   & 0.40                                 & 0.49                                   & 0.49                                             & 0.01                                  \\ \hline
            shell\_script                       & 0.38                                   & 0.50                                 & 0.50                                   & 0.50                                             & 0.04                                  \\ \hline
            python\_wget                        & 0.29                                   & 0.35                                 & 0.39                                   & 0.37                                             & 0.01                                  \\ \hline
            scp                                 & 0.26                                   & $\sim$0.00                                 & 0.07                                   & 0.08                                             & $\sim$0.00                                  \\ \hline
            shell\_wget                         & 0.20                                   & 0.25                                 & 0.18                                   & 0.11                                             & 0.02                                  \\ \hline
            wget                                & 0.21                                   & 0.46                                 & 0.50                                   & 0.50                                             & $\sim$0.00                                  \\ \hline
            \textbf{avg}                                 & 0.25                                   & 0.37                                 & 0.39                                   & 0.32                                             & 0.01                                  \\ \hline
            \end{tabular}
        }
\end{table}

\begin{table}[!tb]
        \centering
        \caption{POI reputation of real attacks (expected: $0.0$)}
        \label{tab:attackLowRP}
        \resizebox{0.48\textwidth}{!}{%
            \begin{tabular}{|l|r|r|r|r|r|}
           \hline
            \multicolumn{1}{|c|}{\textbf{Case}} & \multicolumn{1}{c|}{\textbf{\lpfixed}} & \multicolumn{1}{c|}{\textbf{Fanout}} & \multicolumn{1}{c|}{\textbf{\lpglobal}} & \multicolumn{1}{c|}{\textbf{\lpglobalplus}} & \multicolumn{1}{c|}{\textbf{\tool}} \\ \hline
            data-leakage                        & 0.16                                   & $\sim$0.00                                 & 0.16                                   & 0.15                                             & $\sim$0.00                                  \\ \hline
            password-crack-c1                   & 0.04                                   & 0.25                                 & 0.03                                   & $\sim$0.00                                             & $\sim$0.00                                  \\ \hline
            password-crack-c2                   & 0.40                                   & $\sim$0.00                                 & 0.50                                   & $\sim$0.00                                             & 0.01                                  \\ \hline
            password-crack-c3                   & 0.04                                   & $\sim$0.00                                 & $\sim$0.00                                   & 0.02                                             & $\sim$0.00                                  \\ \hline
            penetration-c1                      & 0.13                                   & 0.30                                 & 0.43                                   & 0.03                                             & 0.03                                  \\ \hline
            penetration-c2                      & 0.13                                   & 0.43                                 & 0.13                                   & 0.13                                             & 0.04                                  \\ \hline
            command-injection-c1                & 0.27                                   & $\sim$0.00                                 & 0.49                                   & $\sim$0.00                                             & $\sim$0.00                                  \\ \hline
            command-injection-c2                & 0.30                                   & $\sim$0.00                                 & 0.24                                   & 0.17                                             & 0.04                                  \\ \hline
            vpnfilter-c1                        & 0.05                                   & $\sim$0.00                                 & 0.02                                   & 0.01                                             & 0.01                                  \\ \hline
            vpnfilter-c2                        & 0.04                                   & $\sim$0.00                                 & $\sim$0.00                                   & $\sim$0.00                                             & $\sim$0.00                                  \\ \hline
            \textbf{avg}                                 & 0.15                                   & 0.10                                 & 0.20                                   & 0.05                                             & 0.01                                  \\ \hline
            \end{tabular}
        \label{tab:attackLowRP}
        }
\end{table}


\myparatight{POI Reputations of \tool}
\cref{tab:benignHighRP,tab:benignLowRP} show the results for benign and malicious payloads through key system interfaces.
\cref{tab:attackLowRP} shows the results for the real attacks.
The results show that \tool effectively propagates the reputation scores from entry nodes to the POI entities (averagely $0.99$ for benign scenarios and averagely $0.03$ in malicious scenarios).
This indicates the effectiveness of our weight computation and reputation propagation.
%inheritance.

\myparatight{Entry Nodes}
The Column ``Entry Nodes'' in \cref{tab:summary} shows the number of entry nodes.
As we can see, most of the entry nodes in dependency graph are libraries nodes, whose reputation can be assigned automatically using a pre-compiled list of verified libraries (for high reputation) and vulnerable libraries (for low reputation).  
The remaining non-lib nodes are likely to be root causes nodes, and their initial reputation could be set by security analysts based on their domain knowledge.
%This result show that
%indicates that 
The results indicate that \tool converts the labor-intensive graph inspection task to the reputation assignment, significantly reducing manual efforts in attack investigation.

\myparatight{Impact of Non-Critical Edges with High Weights} 
Although most of the non-critical edges in a dependency graph have low weights as shown in \cref{fig:box}, it's still possible for some non-critical edges to have high weights (\eg edges corresponding to data exchange of similar amount as the POI in irrelevant activities). 
However, we observe that the path from these activities to the POI are often cut off by edges with very low weights later, and hence these activities will have limited influence on the reputation of the POI node. 
To further confirm our observation, we use DBSCAN~\cite{ester1996density} (epsilon=$0.01$, minimal samples=$5$) to cluster the nodes in the real attack cases based on their final reputations. 
Results show that in all cases the nodes form two clusters and the nodes in the attack path are put in one cluster, which contains only nodes that are relevant to the attack activities.
%In all cases, these clusters contain only nodes relevant to the attack activity, indicating 
This indicates that non-critical edges with high weights did not adversely impact the POI reputation.

\myparatight{Comparisons with Other Weight Computation Approaches}
From \cref{tab:benignHighRP,tab:benignLowRP,tab:attackLowRP},
we have the following observations:
(1) The performance of \lpglobalplus significantly improves over \lpglobal.
This shows the effectiveness and necessity of treating outlier edges differently when doing weight computation;
(2) \lpfixed performs better than \lpglobal and \lpglobalplus in system tasks through key system interfaces in both benign and malicious scenarios (\cref{tab:benignHighRP} and \cref{tab:benignLowRP}).
This shows that the dependency graph is quite diverse and it is difficult to separate all edges into two discriminative groups.
However, treating the outlier edges differently in \lpglobalplus improves over \lpfixed for real attacks;
(3) \tool achieves the best performance in most of the cases. Specifically, \tool achieves an average of 34.67\%, 62.82\%, and 43.76\% improvement over \lpfixed, \lpglobal, and \lpglobalplus in benign scenarios (\cref{tab:benignHighRP}), and an average of 94.52\%, 95.61\%, 86.21\% improvement in malicious scenarios (\cref{tab:benignLowRP,tab:attackLowRP}).
The results clearly show the necessity and superiority of clustering and projecting edges locally for each sink node.
Note that this approach also treats outliers locally by directly setting their weights to 1, and thus \tool embraces the merits of \lpglobalplus and achieves the best performance.


\myparatight{Comparison with Fanout}
The results show that \tool achieves an average of 57.22\% improvement over fanout in benign scenarios (\cref{tab:benignHighRP}) and an average of 87.22\% improvement
%over fanout 
in malicious scenarios (\cref{tab:benignLowRP,tab:attackLowRP}).
As we can see from \cref{tab:attackLowRP}, while the average POI reputation achieved by fanout is $0.10$, it achieves bad performance for \textit{penetration-c1} ($0.30$), \textit{penetration-c2} ($0.43$), and \textit{password-crack-c1} ($0.25$),
which will incorrectly label the malicious payloads as neutral (closing to $0.5$),
while \tool correctly assigns low POI reputations ($\leq 0.04$) for these attack steps.
These results demonstrate the superiority of \tool over fanout.


%\begin{table}[!htb]
\centering
\caption{Seed number for each case}
\begin{tabular}{|l|r|r|}
\hline
Case                    & \multicolumn{1}{l|}{Special Seed} & \multicolumn{1}{l|}{Library} \\ \hline
curl                    & 1                                 & 177                          \\ \hline
mal\_script             & 1                                 & 184                          \\ \hline
python\_wget            & 2                                 & 139                          \\ \hline
scp                     & 1                                 & 52                           \\ \hline
shell\_wget             & 2                                 & 103                          \\ \hline
3File                   & 3                                 & 181                          \\ \hline
2File                   & 2                                 & 180                          \\ \hline
wget                    & 1                                 & 179                          \\ \hline
cmd\_inject\_step3      & 1                                 & 48                           \\ \hline
cmd\_inject\_step5      & 1                                 & 1430                         \\ \hline
shellshock\_leak        & 1                                 & 4246                         \\ \hline
shellshock\_pass\_step1 & 2                                 & 31                           \\ \hline
shellshock\_pass\_step4 & 1                                 & 54                           \\ \hline
shellshock\_pass\_step5 & 4                                 & 14                           \\ \hline
shellshock\_pene\_step1 & 1                                 & 54                           \\ \hline
shellshock\_pene\_step3 & 1                                 & 21                           \\ \hline
vpn\_filter\_step1      & 1                                 & 11                           \\ \hline
vpn\_filter\_step5      & 1                                 & 11                           \\ \hline
\end{tabular}
\label{tab:seedNumber}
\end{table}









\eat{
\myparatight{\tool Summary}\pfang{This part need review}\cref{tab:summary} summarizes the result about benign scenarios and APT attacks. Although the edge number after Backtrack is much smaller compared with the event number in the log, for most cases, it has about 10 thousands to 20 thousands edges. It is still a daunting task for security analysts to identify the critical edges from a graph containing so much edges (\eg \textit{shellshock\_leak}). \tool doesn't only find the critical edges buried by non-critical edges, but also constructs an attack path using these critical edges. At the same time, we may miss some critical edges, because we manually identify these critical edges in the filtered graph provided by \tool. The expected reputation for all cases listed in this table is 0.0. We could see the average reputation of critical nodes is smaller than the non-critical nodes. There are obvious differences between critical and non-critical nodes.

By employing this scheme to compute weights, \tool effectively identifies whether a POI entity comes from trusted or malicious sources by propagating initial reputation from seeds.
}





\eat{
\xiao{XXX} 66.8\% \pfang{: $\frac{1.3467Manual - 0.8072Manual}{0.8072Manual}$ this calculation is based on the average improvement} improvement over the edge priority based on the fanout of nodes. This is calcualted as $\frac{RP_{avg\_LocalProjection} - RP_{avg\_Fanout}}{RP_{avg\_Fanout}}$Compared with the pure edge priority method, \tool also considers time and data information about POI events. The improvement shows the advantage of \tool to process the dependency graph with complex structure. \pfang{need revise}
}


\subsubsection{Graph Reduction Results}
\label{subsec:graphreduction}

\begin{table*}[htb]
	\centering
	\caption{Graph Reduction Result}
	\label{tab:Reduction}
	\resizebox{0.9\textwidth}{!}{%
        \begin{tabular}{|l|r|r|r|r|r|r|r|r|r|r|}
            \hline
            \thead{Case} &\thead{\#N(Original)}&\thead{\#E(Original)}&\thead{\#N(Causality)}&\thead{\#E(Causality)}& \thead{\#N(Merge)}& \thead{\#E(Merge)} & \thead{Node Ratio(\%)}& \thead{Node Reduction(\%)}& \thead{Edge Ratio(\%)} & \thead{Edge Reduction(\%)} \\ \hline
            2File                   &                  75 &             2149 &                   39 &              1900 &             39 &          38 &        52.00 &            48.00 &       1.77 &          98.23 \\\hline
            3File                   &                  78 &             3131 &                   40 &              2815 &             40 &          39 &        51.28 &            48.72 &       1.25 &          98.75 \\\hline
            Python-wget             &                1210 &             9154 &                   73 &              3265 &             73 &          82 &         6.03 &            93.97 &       0.90 &          99.10 \\\hline
            Python-wget-unzip       &                 154 &             7958 &                   86 &              7211 &             86 &         101 &        55.84 &            44.16 &       1.27 &          98.73 \\\hline
            Shell-script            &                  38 &              527 &                   19 &                42 &             19 &          22 &        50.00 &            50.00 &       4.17 &          95.83 \\\hline
            Shell-wget              &                1161 &             9001 &                   31 &              2685 &             31 &          37 &         2.67 &            97.33 &       0.41 &          99.59 \\\hline
            Shell-wget-unzip        &                  93 &             8506 &                   41 &              7705 &             41 &          51 &        44.09 &            55.91 &       0.60 &          99.40 \\\hline
            USB-merge               &                  58 &             5916 &                    8 &              4420 &              8 &          12 &        13.79 &            86.21 &       0.20 &          99.80 \\\hline
            curl                    &                 126 &             3108 &                   61 &              1681 &             61 &          64 &        48.41 &            51.59 &       2.06 &          97.94 \\\hline
            scp                     &                 454 &             3711 &                   58 &               109 &             58 &          74 &        12.78 &            87.22 &       1.99 &          98.01 \\\hline
            wget                    &                  66 &             2457 &                   28 &              2062 &             28 &          30 &        42.42 &            57.58 &       1.22 &          98.78 \\\hline
            command-injection-c1 &                1702 &             6173 &                   51 &                65 &             51 &          51 &         3.00 &            97.00 &       0.83 &          99.17 \\\hline
            command-injection-c2 &                1702 &             6173 &                 1164 &              3417 &           1164 &        1165 &        68.39 &            31.61 &      18.87 &          81.13 \\\hline
            data-leakage            &                2303 &            76478 &                 1809 &             59178 &           1809 &        1818 &        78.55 &            21.45 &       2.38 &          97.62 \\\hline
            password-crack-c1    &                 898 &            44026 &                   35 &               741 &             35 &          36 &         3.90 &            96.10 &       0.08 &          99.92 \\\hline
            password-crack-c2    &                 898 &            44026 &                   60 &             15968 &             60 &          71 &         6.68 &            93.32 &       0.16 &          99.84 \\\hline
            password-crack-c3    &                 898 &            44026 &                   47 &              1055 &             47 &          72 &         5.23 &            94.77 &       0.16 &          99.84 \\\hline
            penetration-c1        &                 375 &             1807 &                   58 &               319 &             58 &          59 &        15.47 &            84.53 &       3.27 &          96.73 \\\hline
            penetration-c2        &                 375 &             1807 &                   25 &               180 &             25 &          86 &         6.67 &            93.33 &       4.76 &          95.24 \\\hline
            vpnfilter-c1         &                 678 &             3076 &                   14 &               274 &             14 &          14 &         2.06 &            97.94 &       0.46 &          99.54 \\\hline
            vpnfilter-c2         &                 678 &             3076 &                   16 &               604 &             16 &          18 &         2.36 &            97.64 &       0.59 &          99.41 \\\hline
            \thead{average}                     &          667.62 &     13632.67 &           179.19 &       5509.33  &     179.19 &   187.62&     27.22 &         72.78 &  2.26 &       97.74 \\\hline
        \end{tabular}
    }
    \dcaption{Graph reduction results after causality analysis and edge merge. Average node reduction (by causality analysis) is 72.78\%. Average edge reduction (by causality analysis and edge merge) is 97.74\%.}
\end{table*}


\cref{tab:Reduction} shows the reduction in the number of nodes and in the number of edges after causality analysis (\cref{subsec:graph-generation}) and edge merge (\cref{subsec:graph-preprocessing}).
As we can see, the reduction is significant: (1) In most of the cases, \tool achieves more than half of nodes reduced. Causality analysis helps trim up to 72.8\% nodes on average.
(2) In most of the cases, \tool achieves more than 95\% of edges reduced. Edge merge helps trim up to 97.74\% edges on average.

To surface critical edges, \tool uses a threshold to hide non-critical edges.
To provide a guidance on selecting this threshold, we test the filtering performance by selecting an increasing multiple of average weight of the whole graph from 0 to 2 with a pace of 0.05.
We define the \emph{threshold} as the average weight of the whole graph magnified by a number $T_w$ (\ie threshold multiplier). 
%
\cref{fig:edge-thresh} shows the average percentage of edges remaining of all cases after filtering. We observe a turning point at $T_w = 0.15$ and the number of remaining edges will remain stable below 20\%. Higher thresholds can lead to more graph size reduction. However, if we choose the threshold too high, we will lose track of some of the critical edges. 
%
We define the \emph{missing point} as the exact threshold multiplier that leads to the first critical edge loss(\cref{tab:filter}). 
\cref{fig:cdf} shows the cumulative distribution of missing points.
We observe that: 
(1) Two cases (\emph{command-injection-c2}, \emph{data-leakage}) have extremely high missing points ($T_w > 200$);
(2) 5 out of 21 cases lost critical edges at $T_w = 2$. However, in these 5 cases, 2 of them(\emph{Shell-wget},\emph{penetration-c1}) already have less than 10 non-critical edges at missing point and 3 of them also have significant reduction in edge numbers(\cref{tab:filter}).
(3) A plateau exists before $T_w = 2$ at a rate of 24\%. This indicate most of the cases have a missing point greater than $T_w = 2$, which proves the efficacy of our weights to differ critical edges from non-critical edges.

Given that setting $T_w = 0.15$ is enough to filter out more than 80\% of the non-critical edges and 76\% of the cases have $T_w > 2$. A good strategy would be examining the graph at $T_w = 2$ to grab a rough sense then tuning down to $T_w = 0.15$ to review details. To avoid critical edge miss in some situation, then going down to $T_w = 0$. Rather than directly examine the graph after Edge Merge, this will save a lot of daunting labor.
  


\begin{table}[]
\centering
        \caption{Filtering Results}
        \label{tab:filter}
        \resizebox{0.45\textwidth}{!}{%
            \begin{tabular}{|l|r|r|r|}
            \hline
            \thead{Attacks} & \thead{\#Critical Edges} & \thead{Missing Point} & \thead{\#Non-critical Edges at Missing Point}\\\hline
            2File                     & 3                        & 9.49          & 0                                     \\\hline
            3File                     & 4                        & 6.49          & 0                                     \\\hline
            Python-wget               & 4                        & 5.19          & 1                                     \\\hline
            Python-wget-unzip         & 8                        & $<0.01$          & 60                                    \\\hline
            Shell-script              & 4                        & 3.15          & 1                                     \\\hline
            Shell-wget                & 4                        & 0.04          & 6                                     \\\hline
            Shell-wget-unzip          & 6                        & 2.83          & 3                                     \\\hline
            USB-merge                 & 6                        & 2.11          & 0                                     \\\hline
            curl                      & 4                        & 12.80         & 1                                     \\\hline
            scp                       & 3                        & 8.29          & 4                                     \\\hline
            wget                      & 2                        & 6.20          & 29                                    \\\hline
            command-injection-c1 & 2                        & 17.00         & 49                                    \\\hline
            command-injection-c2 & 3                        & 286.50        & 0                                     \\\hline
            data-leakage             & 5                        & 302.89        & 0                                     \\\hline
            password-crack-c1    & 2                        & 9.00          & 34                                    \\\hline
            password-crack-c2    & 4                        & 14.20         & 1                                     \\\hline
            password-crack-c3    & 4                        & $<0.01$          & 57                                    \\\hline
            penetration-c1         & 3                        & 0.02          & 5                                     \\\hline
            penetration-c2         & 11                       & $<0.01$          & 21                                    \\\hline
            vpnfilter-c1          & 2                        & 4.67          & 12                                    \\\hline
            vpnfilter-c2          & 3                        & 3.60          & 15                                   \\\hline 
            \thead{average}    & 4                        & 32.92          &14.24 \\\hline
            \end{tabular}
        }
\end{table}
\begin{figure}[hbt!]
    \centering
    \includegraphics[width=0.45\textwidth]{figs/fig:edge-thresh.png}
    \caption{Effectiveness of Filtering}
    \dcaption{The percentage of edges remaining after filtering drops significantly at $T_w = 0.15$ and remains stable below 20\% (\ie filtering threshold equals $T_w$ multiplies the average weight of all edges).}
    \label{fig:edge-thresh}
\end{figure}
\begin{figure}[hbt!]
    \centering
    \includegraphics[width=0.48\textwidth]{figs/fig:cdf.png}
    \caption{Critical Edge Loss from Filtering}
    \dcaption{Missing points distribute mostly between $T_w = 2$ and $T_w = 18$. Note a plateau before $T_w = 2$.}
    \label{fig:cdf}
\end{figure}



