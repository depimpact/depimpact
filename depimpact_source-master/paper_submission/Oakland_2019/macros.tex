%\usepackage{times}
\usepackage{graphicx}
%\usepackage{epsf}
%\usepackage{verbatim}
%\usepackage{psfig}
\usepackage{cite}
\usepackage{url}
\usepackage{color}
\usepackage[dvipsnames]{xcolor}
\usepackage{alltt}
%\usepackage{algorithm2e}
\usepackage{hyperref} %change this later
\hypersetup{
	citebordercolor={0 1 0},
	citecolor=blue,
	colorlinks=false,
	filebordercolor={0 .5 .5},
	filecolor=green,
	linkbordercolor={1 0 0},
	linkcolor=blue,
	menubordercolor={1 0 0},
	pageanchor=true,
	pagebackref=true,
	pdfborder={0 0 1},
	pdfpagelabels=true,
	pdftex,
	plainpages=false,
	urlbordercolor={0 1 1},
	urlcolor=blue,
}
\usepackage{syntax}
\usepackage{newfloat}
\usepackage{multicol}
\usepackage{amsmath}


\usepackage{listings}


\definecolor{mygreen}{rgb}{0,0.6,0}
\definecolor{mygray}{rgb}{0.5,0.5,0.5}
%\definecolor{mymauve}{rgb}{0.58,0,0.82}
\renewcommand{\ttdefault}{pcr}
\renewcommand{\lstlistingname}{Query}

\lstset{
	backgroundcolor=\color{white},
	basicstyle=\scriptsize\ttfamily\bfseries,
	breaklines=true,
	keepspaces=true,
	numbers=left,
	numbersep=4pt,
	numberstyle=\tiny\color{Gray},
	rulecolor=\color{black},
	showspaces=false,
	showstringspaces=false,
	showtabs=false,
	stepnumber=1,
	stringstyle=\color{black},
	tabsize=2,
	language=C++
}
\lstset{
	commentstyle=\color{mygreen},
	frame=lines,
	keywordstyle=\color{OrangeRed},
	keywordstyle=[2]\color{blue},
	morekeywords={with,return,count,distinct,as,at,from,to,before,after,last,forward,backward,sort,by,asc,desc,in}, % no space
	keywords=[2] {proc,file,ip}
}
\newcommand{\incode}[1]{\lstinline{#1}}


\usepackage{subcaption}

\usepackage[labelfont=bf,skip=0pt]{caption}
%\DeclareCaptionType{copyrightbox}
%\captionsetup[figure]{font=bf,skip=0pt}%set figure caption
%\captionsetup[table]{font=bf,skip=0pt}%set table caption



\newcommand{\Fix}[1]{{\large\textbf{FIX}}#1{\large\textbf{FIX}}}
%\newcommand{\Fix}[1]{#1}
\newcommand{\New}[1]{{\large\textbf{New}}#1{\large\textbf{New}}}
\newcommand{\FixResolved}[1]{#1}
\newcommand{\realprojects}{a real project}
\newcommand{\Add}{\CodeIn{add}}
\newcommand{\AVTree}{\CodeIn{AVTree}}
\newcommand{\Assignment}[3]{$\langle$ \Object{#1}, \Object{#2}, \Object{#3} $\rangle$}
\newcommand{\BinaryTreeRemove}{\CodeIn{BinaryTree\_remove}}
\newcommand{\BinaryTree}{\CodeIn{BinaryTree}}
\newcommand{\Caption}{\caption}
\newcommand{\Char}[1]{`#1'}
\newcommand{\CheckRep}{\CodeIn{checkRep}}
\newcommand{\ClassC}{\CodeIn{C}}
\newcommand{\CodeIn}[1]{{\small\texttt{#1}}}
\newcommand{\CodeOutSize}{\scriptsize}
\newcommand{\Comment}[1]{}
\newcommand{\Ensures}{\CodeIn{ensures}}
\newcommand{\ExtractMax}{\CodeIn{extractMax}}
\newcommand{\FAL}{field-ordering}
\newcommand{\FALs}{field-orderings}
\newcommand{\Fact}{observation}
\newcommand{\Get}{\CodeIn{get}}
\newcommand{\HashSet}{\CodeIn{HashSet}}
\newcommand{\HeapArray}{\CodeIn{HeapArray}}
\newcommand{\Intro}[1]{\emph{#1}}
\newcommand{\Invariant}{\CodeIn{invariant}}
\newcommand{\JUC}{\CodeIn{java.\-util.\-Collections}}
\newcommand{\JUS}{\CodeIn{java.\-util.\-Set}}
\newcommand{\JUTM}{\CodeIn{java.\-util.\-TreeMap}}
\newcommand{\JUTS}{\CodeIn{java.\-util.\-TreeSet}}
\newcommand{\JUV}{\CodeIn{java.\-util.\-Vector}}
\newcommand{\JMLPlusJUnit}{JML+JUnit}
\newcommand{\Korat}{Korat}
\newcommand{\Left}{\CodeIn{left}}
\newcommand{\Lookup}{\CodeIn{lookup}}
\newcommand{\MethM}{\CodeIn{m}}
\newcommand{\Node}[1]{\CodeIn{N}$_#1$}
\newcommand{\Null}{\CodeIn{null}}
\newcommand{\Object}[1]{\CodeIn{o}\ensuremath{_#1}}
\newcommand{\PostM}{\MethM$_{post}$}
\newcommand{\PreM}{\MethM$_{pre}$}
\newcommand{\Put}{\CodeIn{put}}
\newcommand{\Remove}{\CodeIn{remove}}
\newcommand{\RepOk}{\CodeIn{repOk}}
\newcommand{\Requires}{\CodeIn{requires}}
\newcommand{\Reverse}{\CodeIn{reverse}}
\newcommand{\Right}{\CodeIn{right}}
\newcommand{\Root}{\CodeIn{root}}
\newcommand{\Set}{\CodeIn{set}}
\newcommand{\State}[1]{2^{#1}}
\newcommand{\TestEra}{TestEra}
\newcommand{\TreeMap}{\CodeIn{TreeMap}}
\newcommand{\smoot}{\textsf{MSeqGen} \xspace}

%\newenvironment{CodeOut}{\begin{scriptsize}}{\end{scriptsize}}
\newenvironment{CodeOut}{\begin{small}}{\end{small}}
\newenvironment{SmallOut}{\begin{small}}{\end{small}}

\newcommand{\pairwiseEquals}{PairwiseEquals}
\newcommand{\monitorEquals}{MonitorEquals}
%\newcommand{\monitorWField}{WholeStateW}
\newcommand{\traverseField}{WholeState}
\newcommand{\monitorSMSeq}{ModifyingSeq}
\newcommand{\monitorSeq}{WholeSeq}

\newcommand{\IntStack}{\CodeIn{IntStack}}
\newcommand{\UBStack}{\CodeIn{UBStack}}
\newcommand{\BSet}{\CodeIn{BSet}}
\newcommand{\BBag}{\CodeIn{BBag}}
\newcommand{\ShoppingCart}{\CodeIn{ShoppingCart}}
\newcommand{\BankAccount}{\CodeIn{BankAccount}}
\newcommand{\BinarySearchTree}{\CodeIn{BinarySearchTree}}
\newcommand{\LinkedList}{\CodeIn{LinkedList}}

\newcommand{\Book}{\CodeIn{Book}}
\newcommand{\Library}{\CodeIn{Library}}

\newcommand{\Jtest}{Jtest}
\newcommand{\JCrasher}{JCrasher}
\newcommand{\Daikon}{Daikon}
\newcommand{\JUnit}{JUnit}

\newcommand{\trie}{trie}

\newcommand{\Perl}{Perl}
\newcommand{\vpnfilter}{VPNFilter\xspace}

\newcommand{\SubjectCount}{11}
\newcommand{\DSSubjectCount}{two}

\newcommand{\Equals}{\CodeIn{equals}}
\newcommand{\Pairwise}{PairwiseEquals}
\newcommand{\Subgraph}{MonitorEquals}
\newcommand{\Concrete}{WholeState}
\newcommand{\ModSeq}{ModifyingSeq}
\newcommand{\Seq}{WholeSeq}
\newcommand{\Aeq}{equality}

\newcommand{\Meaning}[1]{\ensuremath{[\![}#1\ensuremath{]\!]}}
\newcommand{\Pair}[2]{\ensuremath{\langle #1, #2 \rangle}}
\newcommand{\Triple}[3]{\ensuremath{\langle #1, #2, #3 \rangle}}
\newcommand{\SetSuch}[2]{\ensuremath{\{ #1 | #2 \}}}
\newtheorem{definition}{Definition}
\newtheorem{theorem}[definition]{Theorem}
\newcommand{\Equiv}[2]{\ensuremath{#1 \EquivSTRel{} #2}}
\newcommand{\EquivME}{\Equiv}
\newcommand{\EquivST}{\Equiv}
\newcommand{\EquivSTRel}{\ensuremath{\cong}}
\newcommand{\Redundant}[2]{\ensuremath{#1 \lhd #2}}
\newcommand{\VB}{\ensuremath{\mid}}
\newcommand{\MES}{method-entry state}

\newcommand{\SmallSpace}{\vspace*{-1.5ex}}
\newcommand{\Item}{\SmallSpace\item}
\newenvironment{Itemize}{\begin{itemize}}{\end{itemize}\SmallSpace}
\newenvironment{Enumerate}{\begin{enumerate}}{\end{enumerate}\SmallSpace}


\newcommand{\Small}[1]{{\small{#1}}}

\newcommand{\CenterCell}[1]{\multicolumn{1}{c|}{#1}}



%%%
\newcommand{\red}{\textcolor{red}}
\newcommand{\pgao}[1]{\textsf{\color{blue}{({Pgao: #1})}}}
\newcommand{\ting}[1]{\textsf{\color{red}{({Ting: #1})}}}
\newcommand{\pfang}[1]{\textsf{\color{violet}{({Fang: #1})}}}

\newcommand{\argmax}{\operatornamewithlimits{argmax}}
\newcommand{\myparatight}[1]{\smallskip\noindent{\bf {#1}.}}
\newcommand{\norm}[1]{\left\lVert #1 \right\rVert_2}
\newcommand{\sign}{sign}
\newcommand{\eat}[1]{}
\newcommand{\eg}{{\it e.g.,}\xspace}
\newcommand{\ie}{{\it i.e.,}\xspace}

\DeclareMathOperator*{\argmax}{arg\,max}
\DeclareMathOperator*{\argmin}{arg\,min}


\DeclareFloatingEnvironment[
% the file extension for the file used to create the list:
fileext   = logr,% don't use log here!
% the heading for the list:
listname  = {List of Grammars},
% the name used in captions:
name      = Grammar,
% the default floating parameters if the environment is used
% without optional argument:
placement = !htbp
]{BNF}

\captionsetup[figure]{font=bf,skip=0pt}%set figure caption
\captionsetup[table]{font=bf,skip=0pt}%set table caption
\newcommand{\distance}{2pt}
\setlength{\textfloatsep}{\distance}%set distance between figure/tables on the top/bottom with text
\setlength{\floatsep}{\distance}%set distance between figures or tables
\setlength{\intextsep}{\distance}%set distance between figures/tables in text with text
\setlength{\dbltextfloatsep}{\distance} %distance between a figure/table spanning both columns and the text;
\setlength{\dblfloatsep}{\distance} %distance between two figures/tables spanning both columns.

\newcommand*{\thead}[1]{\multicolumn{1}{|c|}{\bfseries #1}} %for table header
\newcommand*{\theadnoline}[1]{\multicolumn{1}{c}{\bfseries #1}} %for table header

\newcommand{\dcaption}[1]{\caption*{\par\small{#1}}}

\usepackage{hyperref}

\usepackage[capitalize,nameinlink]{cleveref}
\hypersetup{%
	bookmarksnumbered, bookmarksopen=true, bookmarksopenlevel=1,%
}




\crefname{figure}{Figure}{Figures}
\crefname{listing}{Query}{Queries}
\crefname{section}{Section}{Sections}
\crefname{table}{Table}{Tables}
\crefname{BNF}{Grammar}{Grammars}
\crefname{algorithm}{Algorithm}{Algorithms}
\crefname{equation}{Equation}{Equations}
