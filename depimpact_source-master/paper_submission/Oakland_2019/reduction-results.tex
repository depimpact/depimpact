\subsubsection{Graph Reduction Results}
\label{subsec:graphreduction}

\begin{table*}[htb]
	\centering
	\caption{Graph Reduction Result}
	\label{tab:Reduction}
	\resizebox{0.9\textwidth}{!}{%
        \begin{tabular}{|l|r|r|r|r|r|r|r|r|r|r|}
            \hline
            \thead{Case} &\thead{\#N(Original)}&\thead{\#E(Original)}&\thead{\#N(Causality)}&\thead{\#E(Causality)}& \thead{\#N(Merge)}& \thead{\#E(Merge)} & \thead{Node Ratio(\%)}& \thead{Node Reduction(\%)}& \thead{Edge Ratio(\%)} & \thead{Edge Reduction(\%)} \\ \hline
            2File                   &                  75 &             2149 &                   39 &              1900 &             39 &          38 &        52.00 &            48.00 &       1.77 &          98.23 \\\hline
            3File                   &                  78 &             3131 &                   40 &              2815 &             40 &          39 &        51.28 &            48.72 &       1.25 &          98.75 \\\hline
            Python-wget             &                1210 &             9154 &                   73 &              3265 &             73 &          82 &         6.03 &            93.97 &       0.90 &          99.10 \\\hline
            Python-wget-unzip       &                 154 &             7958 &                   86 &              7211 &             86 &         101 &        55.84 &            44.16 &       1.27 &          98.73 \\\hline
            Shell-script            &                  38 &              527 &                   19 &                42 &             19 &          22 &        50.00 &            50.00 &       4.17 &          95.83 \\\hline
            Shell-wget              &                1161 &             9001 &                   31 &              2685 &             31 &          37 &         2.67 &            97.33 &       0.41 &          99.59 \\\hline
            Shell-wget-unzip        &                  93 &             8506 &                   41 &              7705 &             41 &          51 &        44.09 &            55.91 &       0.60 &          99.40 \\\hline
            USB-merge               &                  58 &             5916 &                    8 &              4420 &              8 &          12 &        13.79 &            86.21 &       0.20 &          99.80 \\\hline
            curl                    &                 126 &             3108 &                   61 &              1681 &             61 &          64 &        48.41 &            51.59 &       2.06 &          97.94 \\\hline
            scp                     &                 454 &             3711 &                   58 &               109 &             58 &          74 &        12.78 &            87.22 &       1.99 &          98.01 \\\hline
            wget                    &                  66 &             2457 &                   28 &              2062 &             28 &          30 &        42.42 &            57.58 &       1.22 &          98.78 \\\hline
            command-injection-c1 &                1702 &             6173 &                   51 &                65 &             51 &          51 &         3.00 &            97.00 &       0.83 &          99.17 \\\hline
            command-injection-c2 &                1702 &             6173 &                 1164 &              3417 &           1164 &        1165 &        68.39 &            31.61 &      18.87 &          81.13 \\\hline
            data-leakage            &                2303 &            76478 &                 1809 &             59178 &           1809 &        1818 &        78.55 &            21.45 &       2.38 &          97.62 \\\hline
            password-crack-c1    &                 898 &            44026 &                   35 &               741 &             35 &          36 &         3.90 &            96.10 &       0.08 &          99.92 \\\hline
            password-crack-c2    &                 898 &            44026 &                   60 &             15968 &             60 &          71 &         6.68 &            93.32 &       0.16 &          99.84 \\\hline
            password-crack-c3    &                 898 &            44026 &                   47 &              1055 &             47 &          72 &         5.23 &            94.77 &       0.16 &          99.84 \\\hline
            penetration-c1        &                 375 &             1807 &                   58 &               319 &             58 &          59 &        15.47 &            84.53 &       3.27 &          96.73 \\\hline
            penetration-c2        &                 375 &             1807 &                   25 &               180 &             25 &          86 &         6.67 &            93.33 &       4.76 &          95.24 \\\hline
            vpnfilter-c1         &                 678 &             3076 &                   14 &               274 &             14 &          14 &         2.06 &            97.94 &       0.46 &          99.54 \\\hline
            vpnfilter-c2         &                 678 &             3076 &                   16 &               604 &             16 &          18 &         2.36 &            97.64 &       0.59 &          99.41 \\\hline
            \thead{average}                     &          667.62 &     13632.67 &           179.19 &       5509.33  &     179.19 &   187.62&     27.22 &         72.78 &  2.26 &       97.74 \\\hline
        \end{tabular}
    }
    \dcaption{Graph reduction results after causality analysis and edge merge. Average node reduction (by causality analysis) is 72.78\%. Average edge reduction (by causality analysis and edge merge) is 97.74\%.}
\end{table*}


\cref{tab:Reduction} shows the reduction in the number of nodes and in the number of edges after causality analysis (\cref{subsec:graph-generation}) and edge merge (\cref{subsec:graph-preprocessing}).
As we can see, the reduction is significant: (1) In most of the cases, \tool achieves more than half of nodes reduced. Causality analysis helps trim up to 72.8\% nodes on average.
(2) In most of the cases, \tool achieves more than 95\% of edges reduced. Edge merge helps trim up to 97.74\% edges on average.

To surface critical edges, \tool uses a threshold to hide non-critical edges.
To provide a guidance on selecting this threshold, we test the filtering performance by selecting an increasing multiple of average weight of the whole graph from 0 to 2 with a pace of 0.05.
We define the \emph{threshold} as the average weight of the whole graph magnified by a number $T_w$ (\ie threshold multiplier). 
%
\cref{fig:edge-thresh} shows the average percentage of edges remaining of all cases after filtering. We observe a turning point at $T_w = 0.15$ and the number of remaining edges will remain stable below 20\%. Higher thresholds can lead to more graph size reduction. However, if we choose the threshold too high, we will lose track of some of the critical edges. 
%
We define the \emph{missing point} as the exact threshold multiplier that leads to the first critical edge loss(\cref{tab:filter}). 
\cref{fig:cdf} shows the cumulative distribution of missing points.
We observe that: 
(1) Two cases (\emph{command-injection-c2}, \emph{data-leakage}) have extremely high missing points ($T_w > 200$);
(2) 5 out of 21 cases lost critical edges at $T_w = 2$. However, in these 5 cases, 2 of them(\emph{Shell-wget},\emph{penetration-c1}) already have less than 10 non-critical edges at missing point and 3 of them also have significant reduction in edge numbers(\cref{tab:filter}).
(3) A plateau exists before $T_w = 2$ at a rate of 24\%. This indicate most of the cases have a missing point greater than $T_w = 2$, which proves the efficacy of our weights to differ critical edges from non-critical edges.

Given that setting $T_w = 0.15$ is enough to filter out more than 80\% of the non-critical edges and 76\% of the cases have $T_w > 2$. A good strategy would be examining the graph at $T_w = 2$ to grab a rough sense then tuning down to $T_w = 0.15$ to review details. To avoid critical edge miss in some situation, then going down to $T_w = 0$. Rather than directly examine the graph after Edge Merge, this will save a lot of daunting labor.
  


\begin{table}[]
\centering
        \caption{Filtering Results}
        \label{tab:filter}
        \resizebox{0.45\textwidth}{!}{%
            \begin{tabular}{|l|r|r|r|}
            \hline
            \thead{Attacks} & \thead{\#Critical Edges} & \thead{Missing Point} & \thead{\#Non-critical Edges at Missing Point}\\\hline
            2File                     & 3                        & 9.49          & 0                                     \\\hline
            3File                     & 4                        & 6.49          & 0                                     \\\hline
            Python-wget               & 4                        & 5.19          & 1                                     \\\hline
            Python-wget-unzip         & 8                        & $<0.01$          & 60                                    \\\hline
            Shell-script              & 4                        & 3.15          & 1                                     \\\hline
            Shell-wget                & 4                        & 0.04          & 6                                     \\\hline
            Shell-wget-unzip          & 6                        & 2.83          & 3                                     \\\hline
            USB-merge                 & 6                        & 2.11          & 0                                     \\\hline
            curl                      & 4                        & 12.80         & 1                                     \\\hline
            scp                       & 3                        & 8.29          & 4                                     \\\hline
            wget                      & 2                        & 6.20          & 29                                    \\\hline
            command-injection-c1 & 2                        & 17.00         & 49                                    \\\hline
            command-injection-c2 & 3                        & 286.50        & 0                                     \\\hline
            data-leakage             & 5                        & 302.89        & 0                                     \\\hline
            password-crack-c1    & 2                        & 9.00          & 34                                    \\\hline
            password-crack-c2    & 4                        & 14.20         & 1                                     \\\hline
            password-crack-c3    & 4                        & $<0.01$          & 57                                    \\\hline
            penetration-c1         & 3                        & 0.02          & 5                                     \\\hline
            penetration-c2         & 11                       & $<0.01$          & 21                                    \\\hline
            vpnfilter-c1          & 2                        & 4.67          & 12                                    \\\hline
            vpnfilter-c2          & 3                        & 3.60          & 15                                   \\\hline 
            \thead{average}    & 4                        & 32.92          &14.24 \\\hline
            \end{tabular}
        }
\end{table}
\begin{figure}[hbt!]
    \centering
    \includegraphics[width=0.45\textwidth]{figs/fig:edge-thresh.png}
    \caption{Effectiveness of Filtering}
    \dcaption{The percentage of edges remaining after filtering drops significantly at $T_w = 0.15$ and remains stable below 20\% (\ie filtering threshold equals $T_w$ multiplies the average weight of all edges).}
    \label{fig:edge-thresh}
\end{figure}
\begin{figure}[hbt!]
    \centering
    \includegraphics[width=0.48\textwidth]{figs/fig:cdf.png}
    \caption{Critical Edge Loss from Filtering}
    \dcaption{Missing points distribute mostly between $T_w = 2$ and $T_w = 18$. Note a plateau before $T_w = 2$.}
    \label{fig:cdf}
\end{figure}
