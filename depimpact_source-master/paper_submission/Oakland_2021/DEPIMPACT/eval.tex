\section{Evaluation}


We built \tool ($\sim$20K lines of code) upon Sysdig~\cite{sysdig}, and deployed our tool on 2 hosts to collect system auditing events and perform attack investigation. 
The deployed hosts have 12 active users with hundreds of processes, and are used for various types of daily tasks such as file manipulation, text editing, and software development, which are representative of real-world usage. 
During evaluation, the deployed hosts continue to resume their routine tasks to emulate the real-world deployment where irrelevant system activities and attack activities co-exist.
The routine tasks on these machines ensure that enough noise of irrelevant system activities is collected.
We performed a series of attacks based on known exploits~\cite{exploitdb,liu2018priotracker,kwon2018mci,reduction} in the deployed environment, and applied \tool to 
investigate these attacks, demonstrating the effectiveness of \tool.
In total, our evaluations used real system audit logs that contain \emph{100 million} events. 
%we collected \emph{100 million} real system auditing events for evaluation.
%Each attack is done with the time gap being at least 1 hour.

%Specifically, we 
We aim to answer the following research questions:

%\begin{itemize}[noitemsep, topsep=1pt, partopsep=1pt, listparindent=\parindent, leftmargin=*]
\begin{itemize}
\item \textbf{RQ1}: How effective is \tool in revealing attack sequences in comparison with other state-of-art techniques? 
\item \textbf{RQ2}: How do the top-ranked entry nodes affect \tool in revealing attack sequences?
\item \textbf{RQ3}: How effective is \tool in revealing attack entries?
\item \textbf{RQ4}: How efficient is \tool in investigating an attack?
\end{itemize}

% RQ1 aims to measure the effectiveness of using only weights for graph reduction.
% Its results will provide motivation for the design of \tool.
%  Its result evaluate the effectiveness of \tool for dependency graph reduction, which plays an essential role in addressing the challenges mentioned in \cref{sec:intro}.
RQ1 aims to evaluate the overall effectiveness of \tool in dependency graph reduction, and compare \tool with other state-of-the-art causality analysis techniques.
RQ2 aims to evaluate how the top-ranked entry nodes affect the effectiveness of \tool.
RQ3 aims to evaluate whether \tool consistently ranks the attack entries 
%as top-ranked nodes
upfront, and compare \tool with other baseline approaches.
RQ4 aims to measure the execution times of \tool and its variation, and compare \tool with other state-of-the-art causality analysis techniques.

\begin{table*}[!htb]
\centering
\caption{Statistics of dependency graphs generated for the 10 attacks}
\label{tab:stasticalSummary}
\resizebox{0.98\textwidth}{!}{
\begin{tabular}{crrrrrrrr}
\hline
\textbf{Attack}      & \multicolumn{1}{c}{\textbf{Causality Analysis \# V}} & \multicolumn{1}{c}{\textbf{Causality Analysis \# E}} & \multicolumn{1}{c}{\textbf{Edge Merge \#V}} & \multicolumn{1}{c}{\textbf{Edge Merge \# E}} & \multicolumn{1}{c}{\textbf{Entry Nodes}} & \multicolumn{1}{c}{\textbf{Critical Edge}} & \multicolumn{1}{c}{\textbf{Attack Entries}} & \multicolumn{1}{c}{\textbf{POI}} \\ \hline
Wget Executable      & 126.00                                               & 673.00                                               & 126.00                                      & 363.00                                       & 46.00                                    & 8.00                                       & 2.00                                        & 50000000.00                      \\
Illegal Storage      & 8450.00                                              & 93085.00                                             & 8450.00                                     & 62073.00                                     & 960.00                                   & 6.00                                       & 2.00                                        & 50001879.00                      \\
Illegal Storage2     & 42450.00                                             & 658913.00                                            & 42450.00                                    & 378326.00                                    & 3499.00                                  & 4.00                                       & 2.00                                        & 50001879.00                      \\
Hide File            & 194208.00                                            & 6464098.00                                           & 194208.00                                   & 3273769.00                                   & 35203.00                                 & 12.00                                      & 2.00                                        & 50001879.00                      \\
Steal Information    & 195636.00                                            & 6493626.00                                           & 195636.00                                   & 3291208.00                                   & 35213.00                                 & 4.00                                       & 2.00                                        & 50001879.00                      \\
Backdoor Download    & 7510.00                                              & 69479.00                                             & 7510.00                                     & 60390.00                                     & 157.00                                   & 8.00                                       & 2.00                                        & 50000000.00                      \\
Annoying Server User & 114.00                                               & 585.00                                               & 114.00                                      & 318.00                                       & 34.00                                    & 10.00                                      & 2.00                                        & 50000000.00                      \\
Shellshcok           & 81.00                                                & 10289.00                                             & 81.00                                       & 229.00                                       & 107.00                                   & 9.00                                       & 1.00                                        & 124.00                           \\
Dataleak             & 174.00                                               & 734.00                                               & 174.00                                      & 459.00                                       & 49.00                                    & 6.00                                       & 1.00                                        & 7105.00                          \\
VPN Filter           & 549.00                                               & 2986.00                                              & 549.00                                      & 661.00                                       & 507.00                                   & 7.00                                       & 1.00                                        & 1638.00                          \\
Five Dir Case1       & 240.00                                               & 272.00                                               & 240.00                                      & 272.00                                       & 232.00                                   & 2.00                                       & 1.00                                        & 50781.00                         \\
Five Dir Case3       & 5907.00                                              & 78075.00                                             & 5907.00                                     & 78075.00                                     & 879.00                                   & 4.00                                       & 1.00                                        & 121856.00                        \\
Theia Case1          & 184352.00                                            & 816277.00                                            & 184352.00                                   & 816277.00                                    & 151827.00                                & 8.00                                       & 1.00                                        & 166784.00                        \\
Theia Case3          & 334441.00                                            & 1500717.00                                           & 334441.00                                   & 1500717.00                                   & 282651.00                                & 6.00                                       & 2.00                                        & 166640.00                        \\
Trace Case5          & 263.00                                               & 971.00                                               & 263.00                                      & 971.00                                       & 28.00                                    & 3.00                                       & 1.00                                        & 95015.00                         \\
\textbf{AVG}         & 64966.73                                             & 1079385.33                                           & 64966.73                                    & 630940.53                                    & 34092.80                                 & 6.47                                       & 1.53                                        & 23374497.27                      \\ \hline
\end{tabular}
}
\end{table*}

\subsection{Evaluation Setup}
\label{subsec:evalsetup}
To evaluate \tool, we performed 10 attacks in the deployed environment: 7 attacks based on commonly used exploits and 3 multi-step intrusive attacks based on the Cyber Kill Chain framework~\cite{cyberkillchain} and CVE~\cite{cve}.
We then collected system auditing events for the attacks and applied \tool to analyze the events. 
\tool is executed on a server with an Intel(R) Xeon(R) CPU E5-2637 v4 (3.50GHz), 256GB RAM running 64bit Ubuntu 18.04.1.
Next, we describe the attacks in detail.



\subsubsection{Attacks Based on Commonly Used Exploits}
\label{subsub:benign-cases}
These 7 attacks are used as test cases in prior work~\cite{exploitdb,liu2018priotracker,kwon2018mci,reduction},
which consist of the following scenarios: 
%\begin{itemize}[noitemsep, topsep=1pt, partopsep=1pt, listparindent=\parindent, leftmargin=*]
\begin{itemize}
    \item \textit{Wget Executable}: A vulnerable server allows the attacker to download executable files using wget. The attacker downloads python scripts and executes the scripts.
    \item \textit{Illegal Storage}: A server administrator uses wget to download suspicious files to a user's home directory.
    \item \textit{Illegal Storage 2}: A server administrator uses curl to download suspicious files to a user's home directory.
    \item \textit{Hide File}: The goal of the attacker is to hide malicious file among the user's normal files. The attacker downloads the malicious script and hides it by changing its file name and location.
    \item \textit{Steal Information}: The attacker steals the user's sensitive information and writes the information to a hidden file.
    \item \textit{Backdoor Download}: A malicious insider uses the ping command to connect to the malicious server, and then downloads the backdoor script from the server and hides the script by renaming it.
    \item \textit{Annoying Server User}: 
    %A vulnerable server is used by multiple users. 
    The annoying user logs into other user's home directories on a vulnerable server and writes some garbage data to other user's files. 
\end{itemize}


\subsubsection{Real Attacks}
\label{subsubsec:attack-cases}

Besides common exploits, we performed 3 real attacks that capture the important traits of attacks depicted from the Cyber Kill Chain framework~\cite{cyberkillchain}. 
Note that theses attacks consist of a series of steps, and some steps may not be captured by system auditing (\eg user inputs and inter-process communications).
Such limitations can be addressed by employing more powerful auditing tools, which is out of the scope of this paper.
\eat{
\myparatight{Attack 1: Zero-Day Penetration to Target Host}
The scenario emulates the attacker's behavior who penetrates the victim's host
leveraging previously unknown Zero-day attack. Zero-day vulnerabilities are
attack vectors that previously unknown to the community, therefore allow the
attacker to put their first step into their targets. In our case, we assume that
the {\tt bash} binary in victim's host is outdated and vulnerable to shellshock~\cite{shellshock}. The victim computer hosts web service that has
CGI written as BASH script. The attacker can run an arbitrary command when she
passes the specially crafted attack string as one of environment variable. Leveraging the vulnerability, the attacker runs a series of remote commands to
plant and run initial attack by: (1) transferring the payload (\emph{penetration-c1}), (2) changing its permission, and (3) running the payload to bootstrap its campaign (\emph{penetration-c2}).
% As a lateral movement, the
% attacker downloads (4) a password cracker program from outside run it against
% the shadow password files. 
}


\myparatight{Attack 1: Password Cracking After Shellshock Penetration}
% Once breaks into the system, the attacker can launch different malicious
% behaviors (\eg password cracking, information leakage, denial of services). 
After the initial shellshock penetration, the attacker first connects to Cloud services (\eg Dropbox, Twitter) and downloads an image where C2 (Command and Control) host's IP address is encoded in the EXIF metadata (\emph{password-crack-c1}). The behavior is a common practice shared by APT attacks~\cite{hammertoss,vpnfilter} to evade the network-based detection system based on DNS blacklisting.

Using the IP, the malware connects to the C2 host. 
The C2 host directs the malware to take some lateral movements, including a series of stealthy reconnaissance maneuvers. 
In this stage, the attacker generally takes a number of actions. 
Among those, we emulate the password cracking attack. 
The attacker downloads password cracker payload  and runs it against password shadow files (\emph{shellshock}).

\myparatight{Attack 2: Data Leakage After Shellshock Penetration}
After the lateral movement stage, the attacker attempts to steal all the valuable assets from the host. 
This stage mainly involves the behaviors of local and remote file system scanning activities, copying and compressing of important files, and transferring the files to its C2 host.
The attacker scans the file system, scrap files into a single compress file and transfer it back to C2 host (\emph{data-leakage}).

\eat{
\myparatight{Attack 4: Command-line Injection with Input Sanitization Failures}
Different from the previous shellshock case, a program may contain vulnerabilities introduced by developer errors and this can also be a initial attack vector that invites the attacker into their target systems. To represent
such cases, we wrote an web application prototype that fails to sanitize inputs for a certain web request, hence allows Command line Injection attack. 
Our prototype service mimics the Jeep-Cherokee attack case~\cite{miller:remote:2015} which implements a remote access using the conventional web service API that
internally uses DBUS service to run the designated commands. 
Due to the developers' mistake, the web service fails to sanitize the remote inputs, the attacker can append arbitrary commands followed by semi-colon({\tt;}). 
Leveraging this vulnerability, we can download backdoor program (\emph{commend-injection-c1}) and collect sensitive data (\emph{command-injection-c2}).
}

\myparatight{Attack 3: VPNFilter}
We prototyped a famous IoT attack campaign: VPNFilter malware~\cite{vpnfilterschenier}, which infected millions of IoT devices by exploiting a number of known or zero-day vulnerabilities~\cite{vpnfilter1,vpnfilter2}. 
The attack's significance lies in how the malware operates during its lateral movement stage following its initial penetration. 
The campaign employs up-to-date hacker practices to bypass conventional security solutions based on static blacklisting approaches and has an architecture to download the plug-in payload on-demand at run-time. 
We prototyped the malware by referring to one of its sample for x86 architecture~\cite{vpnfilterx86}.

The VPNFilter stage 1 malware accesses a public image repository to get an image. In the EXIF metadata of the image, it contains the IP address for the stage 2 host. It downloads the VPNFilter stage2 from the stage2 server, and executes it to launch the attack (\emph{vpnfilter}).






\myparatight{Dependency Graph Statistics for the 10 Attacks}
\cref{tab:stasticalSummary} shows the statistics of the generated dependency graphs for the 10 attacks. 
Columns ``Causality Ana. \# V'' and ``Causality Ana. \# E'' show the number of nodes and edges after performing the causality analysis from the POI events.
Columns ``Edge Mer. \# V'' and ``Edge Mer. \# E'' show the number of nodes and edges after applying edge merges (\cref{subsubsec:edge-merge}).
Columns ``Entry Nodes'' and  ``Critical Edge'' show the number of entry nodes and critical edges of the dependency graphs. 
Column ``Attack Entries'' shows the number of entry nodes that are attack entries.
%
We clearly observe that even after edge merges, there still remains a large number of edges in the dependency graphs (707K on average with the max being $3.3$ million edges), which motivates the further pruning provided by \tool.




\begin{table}[t]
\centering
\ra{1.2}
\caption{Statistics of dependency graphs generated for the 10 attacks}
\label{tab:rq1}
\resizebox{0.48\textwidth}{!}{
\begin{tabular}{@{}crrrrr@{}}
\toprule
\textbf{Attack}      & \multicolumn{1}{c}{\textbf{CPR}} & \multicolumn{1}{c}{\textbf{ReadOnly}} & \multicolumn{1}{c}{\textbf{PrioTracker}} & \multicolumn{1}{c}{\textbf{NoDoze}} & \multicolumn{1}{c}{\textbf{\tool}} \\ \midrule
Wget Executable      & 363                           & 58                                 & 58                                    & 288                              & 48                                \\
Illegal Storage      & 62,073                        & 16,211                             & 6,948                                 & 10,260                           & 43                                \\
Illegal Storage2     & 378,326                       & 89,779                             & 37,112                                & 19,512                           & 7                                 \\
Hide File            & 3,273,769                     & 613,303                            & 114,614                               & 37,251                           & 437                               \\
Steal Information    & 3,291,208                     & 618,025                            & 115,223                               & 20,426                           & 750                               \\
Backdoor Download    & 60,390                        & 15,990                             & 6,024                                 & 269                              & 20                                \\
Annoying Server User & 318                           & 56                                 & 39                                    & 227                              & 23                                \\
Shellshcok           & 3,600                         & 590                                & 518                                   & 911                              & 451                               \\
Dataleak             & 1,152                         & 231                                & 211                                   & 687                              & 223                               \\
VPN Filter           & 1,879                         & 298                                & 244                                   & 217                              & 263                               \\
Five Dir. Case1      & 272                           & 18                                 & 18                                    & 257                              & 7                                 \\
Five Dir. Case3      & 78,075                        & 77,824                             & 7,496                                 & 598                              & 33                                \\
Theia Case1          & 816,277                       & 325,459                            & 176,800                               & 151,240                          & 62                                \\
Theia Case3          & 1,500,717                     & 537,424                            & 269,277                               & 9,015                            & 10                                \\
Trace Case5          & 971                           & 910                                & 459                                   & 510                              & 4                                 \\
\textbf{AVG}         & 631,292.67                       & 153,078.40                            & 49,002.73                                & 16,777.87                           & 158.73                               \\ \bottomrule
\end{tabular}
}
\end{table}

\begin{table*}[]
\centering
\caption{\tool reduction \& missing result}
\label{tab:toolReductionAndMissing}
\resizebox{\textwidth}{!}{
\begin{tabular}{c|r|r|r|r|r|r|r|r|r}
\hline
\textbf{Case}    & \multicolumn{1}{c|}{\textbf{1 Avg. Missing}} & \multicolumn{1}{c|}{\textbf{1 Avg. Reduction}} & \multicolumn{1}{c|}{\textbf{1 Avg. \# Edge}} & \multicolumn{1}{c|}{\textbf{2 Avg. Missing}} & \multicolumn{1}{c|}{\textbf{2 Avg. Reduction}} & \multicolumn{1}{c|}{\textbf{2 Avg. \# Edge}} & \multicolumn{1}{c|}{\textbf{3 Avg. Missing}} & \multicolumn{1}{c|}{\textbf{3 Avg. Reduction}} & \multicolumn{1}{c}{\textbf{3 Avg. \# Edge}} \\ \hline
wget executable      & 0.31                                               & 0.94                                                 & 21.00                                              & 0.06                                               & 0.91                                                 & 34.27                                              & 0.00                                               & 0.88                                                 & 42.75                                             \\ \hline
illegal storage      & 0.52                                               & 1.00                                                 & 27.67                                              & 0.27                                               & 1.00                                                 & 44.67                                              & 0.16                                               & 1.00                                                 & 55.57                                             \\ \hline
illegal storage 2    & 0.08                                               & 1.00                                                 & 279.67                                             & 0.00                                               & 1.00                                                 & 444.58                                             & 0.00                                               & 1.00                                                 & 536.89                                            \\ \hline
Hide File            & 0.07                                               & 1.00                                                 & 286.89                                             & 0.23                                               & 1.00                                                 & 469.14                                             & 0.17                                               & 1.00                                                 & 588.92                                            \\ \hline
Steal Information    & 0.19                                               & 1.00                                                 & 366.89                                             & 0.02                                               & 1.00                                                 & 591.97                                             & 0.00                                               & 1.00                                                 & 722.38                                            \\ \hline
Backdoor Download    & 0.35                                               & 1.00                                                 & 30.78                                              & 0.11                                               & 1.00                                                 & 52.31                                              & 0.03                                               & 1.00                                                 & 68.79                                             \\ \hline
Annoying Server User & 0.48                                               & 0.97                                                 & 8.50                                               & 0.29                                               & 0.96                                                 & 13.40                                              & 0.23                                               & 0.95                                                 & 16.75                                             \\ \hline
Shellshock           & 0.47                                               & 0.96                                                 & 9.67                                               & 0.25                                               & 0.93                                                 & 14.94                                              & 0.12                                               & 0.92                                                 & 18.57                                             \\ \hline
Dataleak             & 0.53                                               & 0.96                                                 & 19.80                                              & 0.25                                               & 0.93                                                 & 31.50                                              & 0.08                                               & 0.91                                                 & 40.30                                             \\ \hline
Vpnfilter            & 0.24                                               & 0.98                                                 & 12.56                                              & 0.04                                               & 0.97                                                 & 17.42                                              & 0.00                                               & 0.97                                                 & 19.64                                             \\ \hline
\textbf{Avg.}                  & 0.32                                               & 0.98                                                 & 106.34                                             & 0.15                                               & 0.97                                                 & 171.42                                             & 0.08                                               & 0.96                                                 & 211.06                                            \\ \hline
\end{tabular}
}
\end{table*}
\begin{table*}[!htb]
\centering
\caption{Monkey reduction \& missing result}
\label{tab:rq2random}
\resizebox{\textwidth}{!}{
\begin{tabular}{c|r|r|r|r|r|r|r|r|r}
\hline
\textbf{Case}    & \multicolumn{1}{c|}{\textbf{1 Avg. Missing}} & \multicolumn{1}{c|}{\textbf{1 Avg. Reduction}} & \multicolumn{1}{c|}{\textbf{1 Avg. \# Edge}} & \multicolumn{1}{c|}{\textbf{2 Avg. Missing}} & \multicolumn{1}{c|}{\textbf{2 Avg. Reduction}} & \multicolumn{1}{c|}{\textbf{2 Avg. \# Edge}} & \multicolumn{1}{c|}{\textbf{3 Avg. Missing}} & \multicolumn{1}{c|}{\textbf{3 Avg. Reduction}} & \multicolumn{1}{c}{\textbf{3 Avg. \# Edge}} \\ \hline
wget executable      & 0.43                                      & 0.95                                        & 18.93                                     & 0.15                                      & 0.91                                        & 32.10                                     & 0.03                                      & 0.89                                        & 41.29                                    \\ \hline
illegal storage      & 0.32                                      & 1.00                                        & 158.18                                    & 0.19                                      & 1.00                                        & 276.62                                    & 0.17                                      & 0.99                                        & 402.47                                   \\ \hline
illegal storage 2    & 0.08                                      & 0.99                                        & 4,778.19                                  & 0.00                                      & 0.98                                        & 8,777.70                                  & 0.00                                      & 0.97                                        & 12,209.11                                \\ \hline
Hide File            & 0.33                                      & 0.98                                        & 64,486.41                                 & 0.24                                      & 0.97                                        & 113,291.58                                & 0.25                                      & 0.96                                        & 144,893.18                               \\ \hline
Steal Information    & 0.11                                      & 0.98                                        & 62,643.55                                 & 0.01                                      & 0.97                                        & 108,618.02                                & 0.00                                      & 0.96                                        & 143,663.34                               \\ \hline
Backdoor Download    & 0.30                                      & 0.99                                        & 329.58                                    & 0.08                                      & 0.99                                        & 609.42                                    & 0.02                                      & 0.98                                        & 926.98                                   \\ \hline
Annoying Server User & 0.56                                      & 0.97                                        & 10.88                                     & 0.34                                      & 0.94                                        & 18.22                                     & 0.24                                      & 0.92                                        & 24.31                                    \\ \hline
Shellshock           & 0.24                                      & 0.88                                        & 28.50                                     & 0.06                                      & 0.80                                        & 44.66                                     & 0.02                                      & 0.76                                        & 54.99                                    \\ \hline
Dataleak             & 0.21                                      & 0.92                                        & 36.52                                     & 0.04                                      & 0.89                                        & 48.71                                     & 0.00                                      & 0.88                                        & 54.84                                    \\ \hline
Vpnfilter            & 0.27                                      & 0.98                                        & 12.76                                     & 0.06                                      & 0.97                                        & 18.44                                     & 0.01                                      & 0.97                                        & 21.41                                    \\ \hline
\textbf{Avg.}                  & 0.28                                      & 0.96                                        & 13,250.35                                 & 0.12                                      & 0.94                                        & 23,173.55                                 & 0.07                                      & 0.93                                        & 30,229.19                                \\ \hline
\end{tabular}
}
\end{table*}
\subsubsection{Effectiveness of \tool in Graph Reduction}
\label{subsec:reductionAndMissing}

\myparatight{Evaluation Metrics}
To measure the effectiveness of graph reduction and the missing of critical edges, we compute two metrics: $M_{reduction}$ and $M_{missing}$. $M_{reduction}$ is used to measure the graph reduction rate, which is defined as:
\begin{equation}
    M_{reduction} = 1- \frac{N_{critical}}{N_{merge}}
\end{equation}
$N_{critical}$ is the number of edges in the critical component generated by \tool. $N_{merge}$ is the number of edges after preprocessing (\ie EdgeMerge). 
A good reduction method should have a large $M_{reduction}$.

$M_{missing}$ is used to measure the information loss during the graph reduction. 
Because the corresponding information is represented as edges in dependency graph, we use the critical-edge missing rate to measure the attack information loss. 
$M_{missing}$ is defined as:
\begin{equation}
    M_{missing} = \frac{N_{missing}}{N_{total}}
\end{equation}
$N_{missing}$ is the number of missing critical edges in the critical component generated by \tool. Since we have control over the test environment of these attack cases, we are able to figure out the ground truth of the attack sequences. 
We have the total number of critical edges that should be contained by the critical component, which is represented by $N_{total}$.
Without any filtering, the dependency graph constructed via performing a backward causality analysis from a POI has the $M_{missing}$ being $0.0$.



\myparatight{Entry Node Selection}
\tool chooses the top ranked entry nodes to perform forward causality analysis, and filter the edges that do not appear in the forward causality analysis. 
For this evaluation, we choose the top 3 entry nodes of each category as candidates, and thus we will have 9 nodes as candidates. 
Given these candidate nodes, we will assume the users may pick any of these 9 nodes to perform forward causality analysis for reduction.
Thus, we will compute the reduction rates by allowing users to select 1, 2, or 3 nodes among these 9 candidates, respectively, and then compute the average for these rates.

\myparatight{Comparison Approach: Monkey Approach}
To demonstrate the effectiveness of \tool's graph reduction, we compare \tool with the monkey  approach. 
For the monkey  approach, all the entry nodes are candidates and the entry nodes used to do the graph reduction are randomly selected. 
We run the monkey approach for 20 times and compute the average reduction rate.
For fairness, we will compare the results of \tool and the monkey approach with the same number of selected nodes (\ie 1, 2, and 3 nodes), respectively.


% We evaluate the graph reduction and critical-edge missing rate, when the attack investigation uses different number of candidates to do the forward reduction. 

\myparatight{Reduction Result}
% For the dependency graph reduction, if we just simply remove 99.99\% edges of dependency graph, we may have a small graph can be easily analyzed, but it is very possible that we lose all the information about attack. 
% If we only pursue to keep all the attack information, the safest reduction way is just keep all the edges. 
% However, this graph is still too large to be analyzed. 
% The ideal method is try to keep all the critical edges, at the same time remove as much irrelevant edges to POI event as possible.
\cref{tab:toolReductionAndMissing} and \cref{tab:rq2random} show the reduction results.
Columns \emph{1 Avg. Missing}, \emph{1 Avg. Reduction}, and \emph{1 Avg. \# Edge} show the average critical-edge missing rate, the average reduction rate, and the average number of edges in the graph processed by \tool and the monkey approach by only choosing 1 entry node.
Columns \emph{2 Avg. Missing}, \emph{2 Avg. Reduction}, \emph{2 Avg. \# Edge}, \emph{3 Avg. Missing}, \emph{3 Avg. Reduction}, and \emph{3 Avg. \# Edge} show the same metrics, when \tool and the monkey approach use 2 and 3 entry nodes to do the forward causality analysis, respectively. 

As expected, we can clearly see the decrease of critical-edge missing rate from 0.32 to 0.08 for \tool and from 0.28 to 0.07 for the monkey approach by using more entry nodes to do the forward causality analysis for reduction. 
\emph{By using all the top 9 entry nodes, the missing rate can reach $0.0$}.
After the reduction, \tool only has about hundreds of edges, but the graph processed by the monkey approach still has more than 30,000 edges on average. 
Compared with the monkey approach, \tool can further reduce the edges produced by the monkey approach by $99.19\%$, $99.26\%$ and $99.3\%$ using 1, 2 or 3 entry nodes, respectively.
This shows the most of the edges found by the monkey approach are non-critical edges.
Note that this improvement is not at the cost of losing critical edges for attacks. 
If \tool uses 3 entry nodes to perform forward causality analysis for reduction, the critical edge missing rate is 0.08 on average. 
At the same time, the total number of edge in the critical component is $211.06$ on average. 
For the monkey approach, this number is $30,229.19$.

% These results show that \tool is highly effective in graph reduction ($96\%$) by using 3 top ranked entry nodes, producing a dependency graph with about 200 edges on average. 
% At the same time, \tool preserves the attack information by keeping most of the critical edges ($92\%$).
These results demonstrate the superiority of \tool over the monkey approach,
which is mainly due to the better selection of the entry nodes.
For \tool, the selection of entry nodes is based on the relevance score ranking. 
For the monkey approach, this selection is a random decision. 
Based on this finding, we can conclude the effectiveness of the graph reduction highly depends on the selection of entry nodes. 




\begin{table}[t]
\centering
\caption{Average rank of attack entries}
\label{tab:rq3}
\resizebox{0.5\textwidth}{!}{
\begin{tabular}{crrrrr}
\hline
 
\textbf{Attack}        & \multicolumn{1}{c}{\textbf{\tool-{}-}} & \multicolumn{1}{c}{\textbf{\tool-}} & \multicolumn{1}{c}{\textbf{Avg. Proj.}} & \multicolumn{1}{c}{\textbf{Rand.}} & \multicolumn{1}{c}{\textbf{\tool}} \\ \hline
Wget Executable      & 9.50                                                             & 12.50                                                                  & 7.00                                                             & 23.45                                                            & 6.00                                                         \\ 
 
Illegal Storage      & 13.00                                                            & 13.00                                                                  & 7.00                                                             & 475.99                                                           & 3.00                                                         \\ 
Illegal Storage 2    & 1.00                                                             & 1.00                                                                   & 1.00                                                             & 1,893.66                                                         & 1.00                                                         \\ 
 
Hide File            & 15.00                                                            & 10.50                                                                  & 3.00                                                             & 17,284.72                                                        & 1.50                                                         \\ 
Steal Information    & 6.50                                                             & 3.50                                                                   & 3.00                                                             & 17,304.32                                                        & 3.00                                                         \\ 
 
Backdoor Download    & 6.50                                                             & 7.00                                                                   & 7.50                                                             & 76.57                                                           & 6.50                                                         \\ 
Annoying Server User & 2.50                                                             & 4.50                                                                   & 9.50                                                             & 15.82                                                            & 4.00                                                         \\ 
 
Shellshock           & 1.00                                                             & 1.00                                                                   & 1.00                                                             & 22.63                                                            & 1.00                                                         \\ 
Dataleak             & 1.00                                                             & 1.00                                                                   & 1.00                                                             & 48.34                                                            & 1.00                                                         \\ 
 
VPN Filter            & 5.00                                                             & 5.00                                                                   & 4.30                                                             & 236.77                                                           & 4.30                                                         \\ 
\textbf{AVG}         & 6.10                                                             & 5.90                                                                   & 4.43                                                             & 3,738.23                                                         & 3.13                                                         \\ \hline
\end{tabular}
}
\end{table}

%\begin{table}[!t]
\centering
\caption{Performance statistics of \tool}
\label{tab:runtime}
\resizebox{0.48\textwidth}{!}{%
\begin{tabular}{|l|r|r|r|r|r|}
\hline
            \textbf{Case}         & \textbf{Causality} & \textbf{Edge Merge} & \textbf{Node Split} & \textbf{Weight} & \textbf{Rep. Propagation} \\ \hline
penetration-c1       & 0.088                     & 0.001           & 0.002             & 0.013                          & 0.069                              \\ \hline
penetration-c2       & 0.086                     & 0.000           & 0.000             & 0.027                          & 0.001                              \\ \hline
password-crack-c1    & 0.186                     & 0.001           & 0.000             & 0.009                          & 0.000                              \\ \hline
password-crack-c2    & 0.183                     & 0.007           & 0.001             & 0.015                          & 0.001                              \\ \hline
password-crack-c3    & 0.256                     & 0.152           & 0.001             & 0.022                          & 0.000                              \\ \hline
data-leakage         & 0.568                     & 0.450           & 0.019             & 1.697                          & 0.031                              \\ \hline
command-injection-c1 & 0.246                     & 0.001           & 0.001             & 0.020                          & 0.000                              \\ \hline
command-injection-c2 & 0.215                     & 0.025           & 0.006             & 0.668                          & 0.008                              \\ \hline
vpnfilter-c1         & 0.179                     & 0.002           & 0.000             & 0.006                          & 0.000                              \\ \hline
vpnfilter-c2         & 0.162                     & 0.007           & 0.000             & 0.053                          & 0.000                              \\ \hline
\textbf{avg}         & \textbf{0.21}            & \textbf{0.06}  & \textbf{0.003}   & \textbf{0.25}                 & \textbf{0.01}                     \\ \hline
\end{tabular}
}
\end{table}

\begin{table*}[]
\centering
\ra{1.2}
\caption{Runtime performance of \tool and baseline approach}
\resizebox{0.96\textwidth}{!}{
\begin{tabular}{crrrrrrrr}
\hline

                            & \multicolumn{1}{c}{}                                             & \multicolumn{1}{c}{}                                         & \multicolumn{2}{c}{\textbf{Dependency Weight Computation (s)}}                                      & \multicolumn{2}{c}{\textbf{Dependency Impact Propagation (s)}}                                             \\ \cline{4-7} 

\multirow{-2}{*}{\textbf{Attack}} & \multicolumn{1}{c}{\multirow{-2}{*}{\textbf{Causality Ana.(s)}}} & \multicolumn{1}{c}{\multirow{-2}{*}{\textbf{Edge Merge(s)}}} & \multicolumn{1}{c}{\tool} & \multicolumn{1}{c}{Avg. Proj.} & \multicolumn{1}{c}{\tool} & \multicolumn{1}{c}{Avg. Proj.}  & \multicolumn{1}{c}{\multirow{-2}{*}{\textbf{NoDoze(s)}}}\\ \hline
Wget Executable            & 120.97            & 0.05           & 0.26                         & 0.02   & 0.06                  & 0.06     & 1.55\\
Illegal Storage            & 92.86             & 0.38           & 7.43                         & 0.39   & 19.48                 & 47.77    & 173.65\\
Illegal Storage2           & 95.13             & 3.02           & 52.68                        & 33.79  & 160.08                & 1,038.55  & 329.31\\
Hide File                  & 223.63            & 42.16          & 463.68                       & 16.14  & 1,150.35              & 8,486.32  & 899.29\\
Steal Information          & 129.82            & 39.51          & 479.02                       & 15.98  & 1,157.45              & 8,128.28 & 620.87\\
Backdoor Download          & 19.74             & 0.44           & 13.87                        & 0.32   & 12.75                 & 24.05    & 0.71\\
Annoying Server User\_user & 17.23             & 0.01           & 0.18                         & 0.01   & 0.03                  & 0.03     & 0.44\\
Shellshcok                 & 0.05              & 0.03           & 0.07                         & 0.01   & 0.02                  & 0.06     & 0.08\\
Dataleak                   & 0.09              & 0.01           & 0.28                         & 0.02   & 0.14                  & 0.16     & 0.01\\
VPN Filter                 & 0.28              & 0.04           & 0.35                         & 0.03   & 0.11                  & 0.14     & 0.07\\
Five Dir. Case1            & 0.81              & 0.01           & 0.10                         & 0.01   & 0.04                  & 0.02     & 0.02\\
Five Dir. Case3            & 2.38              & 0.29           & 9.68                         & 0.12   & 1.93                  & 2.21     & 39.50\\
Theia Case1                & 73.28             & 8.75           & 276.80                       & 1.88   & 289.77                & 191.85   & 30.36\\
Theia Case3                & 106.17            & 8.34           & 498.81                       & 3.06   & 561.95                & 391.96   & 65.88\\
Trace Case5                & 1.92              & 0.01           & 0.14                         & 0.01   & 0.11                  & 0.01     & 0.54 \\
\textbf{AVG}                        & 58.96             & 6.87           & 120.22                       & 4.78   & 223.62                & 1,220.77  & 144.15\\ \hline
\end{tabular}
}
\label{tab:rq4performance}
\end{table*}


\begin{figure}[t]
    \centering
    \includegraphics[width=0.48\textwidth]{figs/s&p/rq4.pdf}
    \caption{Runtime performance of \tool and NoDoze}
    \label{fig:rq4compare}
\end{figure}

\subsection{RQ4: System Performance}

To understand the performance of \tool, we measure the execution time of each step in \tool, as shown in \cref{tab:rq4performance}.
On average, \tool takes $430s$ to finish
%the processing and computation for one 
analyzing an attack, and dependency graph construction (\ie Causality Analysis) requires $69.98s$ and edge merge requires $8.56s$.
%
We compare \tool with the average-projection approach for dependency weight computation and dependency impact propagation, since they share the same steps for causality analysis and edge merge. 
From \cref{tab:rq4performance}, we observe that 
(1) \tool takes more time for dependency weight computation ($\sim102s$) because \tool uses the Multi-KMeans++ clustering and LDA to find the optimal projection vector;
(2) \tool takes less time for dependency impact propagation. The reason is because the dependency weights computed by \tool are much more discriminative, and hence the score propagation can converge faster.
%but the shorter time for the average-projection approach in dependency weight computation is offsetted by the time needed for score propagation. 
As a result, \tool reduces the execution time by $80.23\%$ when compared with the average-projection approach. 

We also compare the execution time of \tool (dependency weight computation plus dependency impact propagation) with the execution time of NoDoze (anomaly score computation), since they share the same causality analysis and edge merge steps.
\cref{fig:rq4compare} shows the results.
%
We observe that \tool is more efficient than NoDoze for 2 attacks (\ie the ``Illegal Storage'' attack and ``Illegal Storage 2'' attack), as efficient as NoDoze for 5 attacks, and less efficient for 3 attacks.
In particular, while \tool requires more time for processing the 2 attacks whose dependency graphs have more than 3 million edges (\ie the ``Hide File'' attack and the ``Steal information'' attack), \tool produces much smaller graphs ($\sim800$ edges) than NoDoze ($>20,000$ edges).
On average, \tool needs $351.8$s to finish the dependency weight computation and the dependency impact propagation, and NoDoze needs $202.65$s to finish the anomaly score computation ($430.34s$ v.s. $281.19s$ for the whole analysis).
%
Thus, \tool and NoDoze have similar runtime performance for most of the attacks, and NoDoze is more efficient for certain attacks at the cost of poor graph reduction. 



\eat{
Also, the results in \cref{subsec:rq3} show that \tool achieves better ranking for the attack entries than the average-projection approach, and we want to know whether it is at the cost of more computation efforts.
The results are shown in \cref{tab:rq4performance}.



To understand the performance of \tool in investigating real attacks, we measure the execution time of \tool on the attack cases.
\tool starts the computation by parsing a log ($92.252s$ averagely) and building a global graph representation ($3.277s$ averagely).
\cref{tab:runtime} shows the execution time for the remaining components of \tool. 
Besides the steps shown in the preprocessing step, \emph{Causality Analysis}, \emph{Edge Merge}, and \emph{Node split} require $0.21s$, $0.06s$, and $0.003$ on average. 
Note that \emph{Weight Computation} and \emph{Reputation Propagation} only requires $0.25s$ and $0.01s$ on average.
In summary, the total time for running an analysis is about $2$ minutes, but the major cost (\ie log parsing) can be improved by adopting caching or database indexing~\cite{gao2018aiql}.
}


\eat{
\subsection{Evaluation Summary}
The evaluation results show that \tool is effective in preserving all the critical edges and filtering non-critical edges by choosing the top 2 top-ranked entry nodes to do the forward causality analysis for reduction. 
On average, the size of the graph produced by \tool is only $4.9\%$ of the dependency graph generated by directly applying causality analysis from the POI event. 
Such a great reduction result relies on the proper selection of the top-ranked entry nodes for the forward causality analysis. 
The comparison to 4 other state-of-the-art causality analysis techniques show that \tool is at least 33 times more effective in dependency graph reduction, and does not share the same limitations as these techniques (\eg training on an execution profile and reputation assignment).
Our results also show that \tool consistently ranks the ground-truth sources at the top ($3.13$ on average). 
Finally, \tool finishes analyzing an attack within $8$ minutes, which is slightly slower than the state-of-the-art technique NoDoze ($\sim5$ minutes).
Considering the dependency graph has about $130,000$ edges on average, \tool is highly effective and efficient to reveal the critical edges compared with the manual inspection, which will take much more than $8$ minutes.
}




