\begin{abstract}

The need for countering Advanced Persistent Threat (APT) attacks has led to solutions that ubiquitously
monitor system activities in each host, and perform timely attack investigation over the monitoring data.
However, existing solutions require non-trivial efforts of manual inspection, which are labor-intensive and lack of automation.
%and error-prone.
%and thus facing significant limitations in automating the investigation process.
\eat{
However, existing solutions require non-trivial efforts of the security analyst to manually conduct the attack investigation, and thus these approaches face major limitations in automating the investigation process:
(1) causality analysis approaches require that the security analyst manually inspects a huge system dependency graph, 
where nodes represent system entities and edges represent the dependencies among system entities;
%auditing events;
(2) behavior querying approaches require that the security analyst 
%master the query syntax and 
manually constructs behavior queries to search for attack patterns and manually inspects the potentially large query results.
}
%
To bridge the gap,
%In this work, we aim to bridge the gap between the pressing need for automating the attack investigation process and the lack of practical solutions.
%effective and efficient solutions for such purpose.
we propose \tool, a novel approach that facilitates automatic attack investigation via weight-aware reputation propagation from system monitoring.
%
%\tool is based on the key insight that proactively reveals critical edges from non-critical edges and associates critical edges with POI entities for POI diagnosis.
%
Given a POI
%(point-of-interest)
%entity/event 
to be investigated, \tool 
(1) applies causality analysis to construct a system dependency graph for the POI and identify a set of entry nodes, 
(2) adopts novel methods to compute discriminative weights for edges in the dependency graph, so that critical edges that are important to revealing the attack sequence are easily revealed from non-critical edges, and (3) adopts a novel weight-aware reputation propagation scheme to propagate the reputation from 
%seed sources
entry nodes (nodes 
%on the dependency graph 
without incoming edges) to the POI along the weighted edges.
%
The 
%inferred 
POI reputation can be used to identify the root cause nodes among other entry nodes that result in the POI, and determine the trustworthiness/suspiciousness of the POI. 
The discriminative edge weights can be used to reveal critical edges and reconstruct the attack sequence, which details how the POI was created.
\eat{
%nodes 
%of the POI and determine whether the POI originated from trusted sources or untrusted sources.
The discriminative edge weights can be used to surface the critical edges and then reconstruct the attack sequence, which details how the attack was performed.
%which details how the attack starting from the entry points eventually affects the POI.
}
Synergistically, \tool significantly reduces the 
%extensive 
efforts of manual inspection and facilitates automatic investigation.
The evaluation on a wide range of 
%attacks on key system interfaces and real
attacks demonstrates the practical efficacy of \tool.
%in automating the attack investigation process.




\end{abstract}